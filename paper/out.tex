\section{Introduction} % (fold)
\label{sec:introduction}

Wide binary stars are valuable tools for studying star and planet formation as well
as Galactic dynamics and chemical evolution.
In the context of studying the evolution of the Milky Way, they are useful for
two main reasons.
First, because wide binaries are weakly bound systems that may be tidally
disrupted by, e.g., field stars, molecular clouds, or the Galactic tidal field,
their statistics can be informative of the Galactic mass distribution.
For example, the separation distribution of halo binaries has been used to
constrain the mass of massive compact halo objects (\citealt{Yoo:2004aa}).
On the other hand, they are also prime targets to test the chemical tagging
hypothesis that stars from the same birthplace may be traced back using detailed
chemical abundance patterns as birth ``tag''s (\citealt{2002ARA&A..40..487F}).
A multiple-star system of any size that we believe are born together can serve
this purpose, with massive open clusters at one extreme.
Wide binaries are at the other extreme in that we may be less confident of the
coevality of individual systems but they are much more abundant than larger
systems like open clusters, rendering their statistics a meaningful indication
of whether the hypothesis works.

Wide binaries with similar or very different planetary architectures around the
component stars have been used to connect a host star's chemistry to its
planetary system.
A differential analysis may reveal chemical signatures related to planet
formation or accretion without any regard to Galactic chemical evolution as
long as the two stars are assumed to be born together with identical
composition.
For close-in giant planets, it has been established that the planet-metallicity
relation is primordial, and that the enhanced metallicity in giant
planet-hosting stars likely promotes the formation of giant planets by
increasing the availability of small particle condensates that form
planetesimals (\citealt{Fischer:2005aa}).
On the other hand, if host stars are polluted after their birth by rocky
planetary material with high refractory-to-volatile ratio, the convective
envelope of the stars may be enhanced in refractory elements (e.g., \elem{Fe})
compared to their initial state.
Thus, differences in planet formation or accretion in two otherwise identical
stars may imprint differences in chemical abundances that depend on the
condensation temperature (\Tcondens).

Several systems in which at least one star hosts at least one planet
have been studied so far with high resolution spectroscopy in this context
(\citealt{Teske:2013aa,Mack:2014aa,Liu:2014aa,Teske:2015aa,Saffe:2015aa,
  Ramirez:2015aa,Biazzo:2015aa,Mack:2016aa,Teske:2016aa,Teske:2016ab}).
The results and interpretations of these studies are varied:
while some systems appear to have undetectable differences
(see also \citealt{Desidera:2004aa,Gratton:2001aa}), other
studies have reported a \Tcondens-dependent difference in abundance
with higher-\Tcondens\ elements showing larger differences.
The increasing abundance difference with condensation temperature has been
attributed both to more yet-undiscovered rocky planets around the relatively
refractory-poor star and to late time accretion of refractory-rich planetary
material (\citealt{Ramirez:2015aa,Biazzo:2015aa}).
The observed differences are $\lesssim 0.1$~dex even in the most dramatic case,
and often at $\approx 0.05$~dex level making them challenging to detect even
with a careful analysis of high-resolution high signal-to-noise ratio spectra.
% maybe want to point out the former interpretation roots back to melendez+09
% and some doubt to that interpretation...

So far the strongest evidence for accretion of planetary material by a host
star comes from spectral analysis of polluted white dwarfs (WDs).
Because the gravitational settling times of elements heavier than \elem{He} in
atmosphere is much shorter than the WD cooling time, detection of metals likely
indicates the presence of a reservoir of dusty material around the WD.
Indeed, many of the polluted WDs host a dusty debris disk detected
in the infrared.
Some of the most dramatically polluted WDs show
surface abundances closely matched by rocky planetary material
with, e.g., bulk Earth composition, strongly arguing
that the disk formed from tidally disrupted minor planets
(\citealt{Zuckerman:2007aa,Klein:2010aa}).
Recently, transit signals from small bodies orbiting around a polluted WD
have been detected by \project{Kepler} adding further support to the picture
(\citealt{2015Natur.526..546V}).



\subsection{Review of Detailed Chemical Abundance Studies of Wide Binary Stars}
\label{sub:review}

Anticipating the forthcoming discussion, we review and summarize a handful of
wide binary systems that have been studied in
their detailed chemical abundances so far with high-resolution spectroscopy.
These systems are HD~20782/HD~20781, HD~80606/HD~80607, XO-2N/XO-2S, HAT-P-1,
WASP-94A/WASP94-B, and HD~133131A/HD~133131B.
We focus on key characteristics of stars and planets, and interpretations of
any trend in $\Delta\elemH{X}$ with \Tcondens.
(\todo{smoh: discussion of HD~1333131AB and 16 Cyg is missing, but will be added})

{\bf HD~20782/HD~20781:}
Two common proper motion G dwarf stars (G2/G9.5) with a projected separation of
$\sim9000$~AU (corresponding to 4.2\arcmin\ sky separation) and solar metallicity
host close-in giant planets.
HD~20782 hosts a Jupiter-mass planet on a very eccentric ($e\approx 0.97$)
orbit with a pericenter distance of 1.4~AU while HD~20781 hosts two
Neptune-mass planets within 0.3~AU with moderately high eccentricity
($e\sim0.1-0.3$).\footnote{
  The two stars were monitored by \project{HARPS} campaign, and it has recently
  been reported by \citealt{2017arXiv170505153U} that HD~20781 hosts four
  planets between $M\sin(i)\approx 0.006-0.04$~\mjupiter\ with $e \le 0.11$
  within $\approx 0.35$~AU.}
The measured abundances of 15 elements in the two stars are consistent
(\citealt{Mack:2014aa}).
However, \citealt{Mack:2014aa} argued that there is a moderately significant
($\sim 2\sigma$) positive slope of $\approx 10^{-5}$~dex\,K$^{-1}$ with
increasing \Tcondens\ for $\Tcondens>900$~K elements (namely, Na, Mn, Cr, Si,
Fe, Mg, Co, Ni, V, Ca, Ti, Al, Sc leaving out C and O of their measurements) in
the abundances of each star {\it individually}.
They suggest that this slope is evidence that the stars accreted
$10-20$~\mearth\ or \elem{H}-depleted rocky material during giant planet
migration.

{\bf HAT-P-1:}
A pair of G0 stars separated by 11\arcsec\ with $\feh\approx0.15$ has
different planetary systems:
the secondary star is known to host one transiting giant planet
while no planet has been discovered around the primary star.
The two stars are identical
in metallicities and abundances for 23 elements measured with
the mean error of $0.013$~dex (\citealt{Liu:2014aa}).
Thus it seems that the presence of close-in giant planet does not necessarily
lead to atmospheric pollution of its host star.

{\bf HD~80606/HD~80607:}
Similar to HAT-P-1, no significant chemical difference is found between two
common proper motion G5 stars with super-solar metallicity ($\feh \approx
0.35$): HD~80606 which hosts a very eccentric ($e\approx0.94$) giant planet and
HD~80607 which has no detected planets (\citealt{Saffe:2015aa,Mack:2016aa}).

{\bf XO-2N/XO-2S:}
A pair of G9 stars with super-solar metallicity ($\feh \gtrsim 0.35$)
has received much attention due to possibly the most robust and significant
difference between the component stars.
XO-2N hosts a giant planet while XO-2S is known to host two giant planets with
masses $0.26 \mjupiter$ and $1.37 \mjupiter$ on moderately eccentric ($\approx
0.15$) orbits at $<0.5$~AU.
%TODO: check following with Biazzo+
All measured elements including \elem{Fe} are enhanced in XO-2N relative to
XO-2S, and the difference shows a significant correlation with \Tcondens\
(\citealt{Teske:2015aa,Ramirez:2015aa,Biazzo:2015aa} although see also
\citealt{Teske:2013aa}).
At low \Tcondens, volatile elements differ by $0.015$~dex while the range of
difference spans upto $0.1$~dex at $\Tcondens>1600$~K.

\citealt{Ramirez:2015aa} argued that the {\it depletion} of volatile elements
in XO-2S relative to XO-2N is plausibly due to the presence of {\it more} gas
giant planets around XO-2S, following a similar interpretation of
\citealt{Melendez:2009aa} of the trend between solar twins and the Sun.
In this scenario, forming planets in the protoplanetary disk are supposed to
``lock'' heavier elements to the core of gas giant planets as well as some
volatile elements in their envelopes. Then, the host star will accrete {\it
  gas} depleted in metals compared to their protostellar condition.
By tuning the metal content ($Z/X$) of gas giant planets and when the accretion
of gas occurs in terms of how much mass is in the convective envelope of the
host star, they can match the mass difference of gas giant planets around each
star, which is at least $1~\mjupiter$.
On the positive correlation of $\Delta\elemH{X}$ with \Tcondens,
\citealt{Ramirez:2015aa} estimated that $20~\mearth$ of equal-part mixture of
Earth and CM chondrite-like material can explain the trend either as refractory
{\it depletion} in XO-2S due to {\it more} rocky planets around XO-2S, or
refractory enhancement in XO-2N by late time accretion in which case the
opposite would be true.

{\bf WASP-94A/WASP-94B:}
Each star in a pair of F8 and F9 stars with super-solar metallicity
($\feh\approx 0.3$) hosts a hot Jupiter.
The planet around WASP-94A is transiting with a misaligned, probably retrograde
circular ($e<0.13$) orbit, while that hosted by WASP-94B is a little more
massive by $\sim 0.15$~\mjupiter\ and closer in, aligned with the host star.
WASP-94A shows a depletion of $0.02$~dex in volatile and moderately volatile
elements ($\Tcondens < 1200$~K) and an enhancement of $0.01$~dex in refractory
elements ($\Tcondens>1200$~K) relative to WASP-94B, with a claimed median
uncertainty of $0.006$~dex among all elements
resulting in a statistically significant non-zero slope between
$\Delta\elemH{X}$ and $\Tcondens$ (\citealt{Teske:2016aa}).\footnote{
  %TODO: double check the following statement
  Note that the condensation
  temperature $\Tcondens$ used is for solar system composition
  gas, which can differ from that of higher metallicity gas.}
Multiple possibilities related to the formation and evolution
of planetary systems around each star as well as causes unrelated to planets
such as dust cleansing during the fully convective phase or different rotation
and granulation between the stars were considered, but none was favored.


% Section: Data

We refer the readers to this previous work (\citealt{2017AJ....153..257O}) for a
full explanation of the methodology behind this search and only briefly describe
the method here.
For a given pair, we compute a marginalized likelihood ratio between the
hypotheses (1) that a given pair of stars share the same 3D velocity vector, and
(2) that they have independent 3D velocity vectors, both given only observations
of two components of the velocities (parallaxes and proper motions).
We then select a sample of high-confidence comoving star pairs by making a
conservative cut on this likelihood ratio.
In the resulting catalog of comoving pairs (\citealt{2017AJ....153..257O}),
the pair presented in this paper was assigned a group id of 1199,
and the marginalized likelihood ratio (Bayes factor)
between the two hypotheses is $\ln{\mathcal{L}_1/\mathcal{L}_2} = 8.52$,
well above the adopted cut value of 6.
We have checked that we do not find any possible additional comoving companions
by lowering the likelihood ratio cut for the stars around this pair.

We start by generating posterior samples over the Heliocentric distance, $r$,
tangential velocities, $(v_\alpha, v_\delta)$, and radial velocity, $v_r$,
given the observed parallax, $\hat\pi$, and proper motions,
$(\hat\mu_{\alpha^*}, \hat\mu_\delta)$, and radial velocity, $\hat v_r$.
We assume the noise is Gaussian, and the radial velocity measurement is
uncorrelated with the astrometric measurements.
If we define
\begin{eqnarray}
  \vec{\hat y} &=&
      \transp{\left(
        \begin{array}{c@{\hspace{1em}} c@{\hspace{1em}} c@{\hspace{1em}} c}
          \hat\pi &
          \hat\mu_{\alpha^*} &
          \hat\mu_\delta &
          \hat v_r
        \end{array}
      \right)}\\
  \vec{y} &=&
      \transp{\left(
        \begin{array}{c@{\hspace{1em}} c@{\hspace{1em}} c@{\hspace{1em}} c}
          r^{-1} &
          r^{-1}\,v_\alpha &
          r^{-1}\,v_\delta &
          v_r
        \end{array}
      \right)}
\end{eqnarray}
then the likelihood is
\begin{equation}
  \vec{\hat y} \sim \mathcal{N}(\vec{y}, \mat{C})
\end{equation} where $\mat{C}$ is the covariance matrix.
We adopt a uniform space density prior for the distance and an isotropic
Gaussian for any velocity component, $v$, with a dispersion $\sigma_v=25~\kms$
\begin{eqnarray}
p(r) &=&
  \begin{cases}
    \frac{3}{r_{\rm lim}^3} \, r^2 & \text{if } 0 < r < r_{\rm lim}\\
    0              & \text{otherwise}
  \end{cases}\\
p(v) &=& \frac{1}{\sqrt{2\pi}\,\sigma_v} \,
  \exp\left[-\frac{1}{2} \, \frac{v^2}{\sigma_v^2} \right] \quad .
\end{eqnarray}
%
For each of the two stars, we use \project{emcee}
(\citealt{2013PASP..125..306F}) to generate posterior samples in $(r, v_\alpha,
v_\delta, v_r)$ by running 64 walkers for 4608 steps and discarding the first
512 steps as the burn-in period.
For each sample, we convert the heliocentric phase-space coordinates into
Galactocentric coordinates assuming that the Sun's position and velocity are
$\vec x_\odot = (-8.3,\,0,\,0)~{\rm kpc}$ and $\vec v_\odot =
(-11.1,\,244,\,7.25)~\kms$ \citep[e.g.,][]{Schonrich:2010, Schonrich:2012}.

\figurename~\ref{fig:dxdv} shows differences in posterior samples converted to
Galactocentric phase-space coordinates for the two stars.
The differences in positions and velocities are consistent with zero.
\todo{should we instead show $|\Delta\boldsymbol{x}|$ and
$|\Delta\boldsymbol{v}|$ computed from the 6D posterior samples?}


% Section discussion

We discuss the possible origins of the peculiar abundance differences of
\bizarreone-\sunanalog.
We first discuss the ages and coevality of the stars in this pair, and consider
both possibilities in which the two stars are or are not coeval.
Our favored scenario is discussed in the last subsection,
\sectionname~\ref{sub:accretion}.

% chance pair
%Given that their metallicities and abundance patterns are significantly
%different, one may simply conclude that the two stars are not related (coeval)
%but they merely happen to be comoving at such small separation ($\approx 0.6$~pc)
%by chance.
%The two stars are in the Galactic disk, and assuming certain velocity ellipsoid
%at the stars' location, one may compute the probability that a star moving at
%the mean velocity of the two stars would have a companion within $\Delta v = XX$~km/s.
%...

\subsection{Exchange Scattering}
\label{sub:exchange_scattering}

Although the data described above strongly suggests that the two stars are
coeval, we may still consider scenarios in which the two stars are not actually
born together.
Two stars unrelated at birth may end up in a binary system via a binary-single
scattering event that results in an exchange of binary members.
In order to estimate the rate at which any binary-single
event will produce a wide binary system such as \sunanalog\ and \bizarreone,
we may consider the rate at which this wide binary will scatter with a field star to
result in an exchange reaction.
The cross-section of exchange scattering for a binary with semi-major axis $a$ is
\begin{eqnarray}
  \sigma_\mathrm{ex} = \frac{640}{81} \pi a^{2} \left(\frac{v_i}{v_c}\right)^{-6}
  \label{eq:crosssection}
\end{eqnarray}
where $v_i$ is the incoming velocity, and $v_c$ is the critical velocity,
defined as
\begin{eqnarray}
  v_c^2 = G \frac{m_1 m_2 (m_1 + m_2 + m_3)}{m_3 (m_1 + m_2)} \frac{1}{a}\,\,.
\end{eqnarray}
\eqname~\ref{eq:crosssection} is appropriate when $v \gg 1$
(\citealt{Hut:1983aa,Hut:1983ab}), which is the case for wide binaries
scattering with field (disk) stars.
If we assume that field stars are made of solar mass stars with a constant
number density $n=1$~pc$^{-3}$, and the incoming velocity of field stars is
$10$~km\,s$^{-1}$, a lower limit to the velocity dispersions of disk stars in
any direction, the rate of exchange scattering is
\begin{eqnarray}
  n \sigma_\mathrm{ex} v_i = 6.82\times 10^{-8}\,\mathrm{Gyr}^{-1}
  \frac{n}{\mathrm{pc}^{-3}} \frac{\mathrm{pc}}{a}
  \left(\frac{10~\mathrm{km}\,\mathrm{s}^{-1}}{v_i}\right)^5\,,
\end{eqnarray}
which is low enough to be negligible.

In fact, any exchange scattering scenario is unsatisfactory as typical
field stars with similar metallicities as \bizarreone\ and \sunanalog\ is very
unlikely to show the abundance difference pattern observed.
In \figname~\ref{fig:deltaXH}, we compare the abundance difference of
\bizarreone-\sunanalog\ with the distribution of abundance differences,
$\Delta[\elem{X}/\elem{H}]$, between random pairings of two stars with similar
\feh\ as \bizarreone\ and \sunanalog.
We see that when a star is enhanced in $\elem{Fe}$ by $0.2$~dex,
all other elements are typically enhanced at a similar level, with some variations.
Specifically, for a typical star with $\feh \approx 0.2$~dex, we generally
expect $[\elem{Na}/\elem{Fe}] > 0$ and $[\elem{Mn}/\elem{Fe}] > -0.1$
(\citealt{Battistini:2015aa,Bensby:2003aa}) making the low
[\elem{Na}/\elem{Fe}] and [\elem{Mn}/\elem{Fe}] seen in \bizarreone\ very
unlikely to arise from variations in Galactic chemical evolution.

\todo{HOGG: to provide some theoretical account adding to the above empirical evidence.}

%TODO: Implications
%- chemical tagging, frequency is of concern
%- exoplaneteers looking for accretion signature: look for the most odd ones in
%the expected direction not just in binaries already known to host a planet.



\section{Summary}
\label{sec:summary}

We report the discovery of a comoving pair of bright
solar-type stars HD~240430 and HD~240429 (G0 and G2) with very different
metallicities ($\Delta\feh \approx 0.2$~dex), and condensation temperature
(\Tcondens)-dependent abundance differences.
The more metal-rich of the two stars, HD~240430 (\bizarreone), shows enhancement in
all ten elements with $\Tcondens > 1200$~K including \elem{Fe}, while under
enhanced in the five elements, \elem{C}, \elem{N}, \elem{O}, \elem{Na}, and
\elem{Mn} with $\Tcondens < 1200$~K relative to HD~240429 (\sunanalog).
The two stars also have anomalously high surface \elem{Li} abundance compared
to their ages of $\sim 4$~Gyr.
We consider that the comoving pair may have formed from two stars of different
birth origins in a exchange scattering event, or that there may be chemical
inhomogeneity in the birth cloud, and find both unlikely.

In order to explain the $\Tcondens$-dependent enhancement and high \elem{Li}
abundance, we consider accretion of planetary material as the cause.
We argue that a recent accretion of $20$~\mearth\ of bulk Earth
composition to \bizarreone\ can explain the enhancement in both refractory
elements and lithium.
For \sunanalog\ which also has high surface \elem{Li} abundance given its age,
we put an upper limit of $\approx 4$~\mearth\ on the accreted mass
based on the \elem{Li} concentration of carbonaceous chondrites and bulk Earth.
While the case is not as clear as \bizarreone\ without a chemical reference star,
the range of abundances from volatile to refractory elements is
similar to that expected from $\approx 4$~\mearth\ accretion of bulk Earth.
It is unclear what triggered the accretions to both stars.
One possibility is that a fly-by field star may have
perturbed the planetary systems of both stars.

