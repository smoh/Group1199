\documentclass[manuscript]{aastex6}

% to-do list
% ----------
% - add items here
% style notes
% -----------
% - This file generates by Makefile; don't be typing ``pdflatex'' or some bullshit.
% - Line break between sentences to make the git diffs readable.
% - Use \, as a multiply operator.
% - Reserve () for function arguments; use [] or {} for outer shit.
% - Use \sectionname not Section, \figname not Figure, \documentname not Article or Paper or paper.

\include{gitstuff}
\include{aastexmods}

% packages
\definecolor{cbblue}{HTML}{3182bd}
\usepackage{microtype}  % ALWAYS!
\usepackage{amsmath,amssymb}
\usepackage{tikz}
\hypersetup{backref,breaklinks,colorlinks,urlcolor=cbblue,linkcolor=cbblue,citecolor=black}
\graphicspath{{figures/}}

% define macros for text
\newcommand{\project}[1]{\textsl{#1}}
\newcommand{\acronym}[1]{{\small{#1}}}
\newcommand{\gaia}{\project{Gaia}}
\newcommand{\rave}{\project{\acronym{RAVE}}}
\newcommand{\apogee}{\project{\acronym{APOGEE}}}
\newcommand{\tmass}{\project{\acronym{2MASS}}}
\newcommand{\documentname}{Article}
\newcommand{\sectionname}{Section}
\newcommand{\figname}{Figure}
\newcommand{\eqname}{Equation}
\newcommand{\dr}{\acronym{DR1}}
\newcommand{\tgas}{\acronym{TGAS}}
\newcommand{\etal}{\textit{et al}.}
\newcommand*\elem[1]{\ensuremath{\mathrm{#1}}}



% define macros for math
\newcommand{\given}{\,|\,}
\newcommand{\normal}{{\mathcal{N}}}
\newcommand{\dd}{\mathrm{d}}
\newcommand{\transp}[1]{{#1}^{\!\mathsf{T}}}
\newcommand{\inv}[1]{{#1}^{-1}}
\newcommand{\bs}[1]{\boldsymbol{#1}}
\newcommand{\vperp}{\bs{v}^\perp}
\newcommand{\propm}{\bs{\mu}}
\newcommand{\mat}[1]{\mathbf{#1}}
\renewcommand{\vec}[1]{\bs{#1}}
\newcommand{\kms}{\ensuremath{\rm km~s^{-1}}}
\newcommand{\msun}{{\rm M}_\odot}
\newcommand{\data}{\mathrm{data}}
\newcommand{\snr}{[S/N]_\varpi}
\newcommand{\eye}{\mathbb{I}}
\newcommand{\absdvtan}{\ensuremath{|\Delta\vec v_\mathrm{t}|}}
\newcommand{\estimates}{\ensuremath{\{\hat{\varpi_i},\hat{\mu_{\alpha,i}},\hat{\mu_{\delta,i}},\hat{v_{r,i}}\}}}

\newcommand{\todo}[1]{{\color{blue}TODO:#1}}

\begin{document}\sloppy\sloppypar\raggedbottom\frenchspacing % trust me

\title{NEED A BETTER TITLE: A co-moving pair of bright stars \\
  with very different metallicities and abundances}

\author{
  Semyeong Oh\altaffilmark{\pu,\lead},
  APW, JMB, KH, DWH, DNS
}

% Affiliations
\newcommand{\pu}{1}
\newcommand{\lead}{2}
\newcommand{\ccpp}{3}
\newcommand{\mpia}{4}
\newcommand{\cca}{5}

\altaffiltext{\pu}{Department of Astrophysical Sciences,
                   Princeton University, Princeton, NJ 08544, USA}
\altaffiltext{\lead}{To whom correspondence should be addressed:
                     \texttt{semyeong@astro.princeton.edu}}
\altaffiltext{\ccpp}{Center for Cosmology and Particle Physics,
                     Department of Physics,
                     New York University, 4 Washington Place,
                     New York, NY 10003, USA}
\altaffiltext{\mpia}{Max-Planck-Institut f\"ur Astronomie,
                     K\"onigstuhl 17, D-69117 Heidelberg, Germany}
\altaffiltext{\cca}{Center for Computational Astrophysics, 162 5th Ave, New York, NY 10003, USA}


\begin{abstract}
  We report and discuss the discovery of a co-moving pair of bright stars,
  HD 240429 and HD 240430, with their detailed abundance measured.
  The kinematics of the two stars from Gaia proper motions and spectroscopic RVs
  are completely consistent with them co-moving in space.
  Yet, the metallicities and the abundances of XX elements between the stars
  separated by $\sim 0.6$~pc in space are very different,
  challenging the simple picture that these co-moving pairs are once wide binaries
  formed in a coeval star-forming event but now tidally stripping.
  We consider such and such scenarios of the formation of such pairs.
\end{abstract}

\section{Introduction} % (fold)
\label{sec:introduction}

% section introduction (end)

\section{Data}
\label{sec:data}

In this section, we describe the data relevant to our discussion of the origin of the system.
The co-moving pair was initially discovered in our recent search
for co-moving pairs using the proper motions from Gaia DR1.
We refer the readers to \citealt{2016arXiv161202440O} and briefly describe the method here.
For a given pair, we performed a hypothesis test between the two hypotheses that
the two stars share the same 3D velocity vector and that they have
two independent 3D velocity vectors.
In the catalog of co-moving pairs (\citealt{2016arXiv161202440O}),
the pair that we present here was assigned a group id of 1199,
and the marginalized likelihood ratio (Bayes factor)
between the two hypotheses is $\ln{\mathcal{L}_1/\mathcal{L}_2} = 8.52$,
well above a relatively conservative cut of 6 that we adopt.
Additionally, we have checked that we do not find any possible additional co-moving companions
to this pair even when we lower the selection threshold to less conservative but still reasonable values.
The pair is in fact classified as a double star system independently
in Washington Double Star Catalog (\cite{2001AJ....122.3466M}).

In a separate effort to study detailed chemical abundances of
potential planet-hosting stars, \citet{2016ApJS..225...32B} studied these two stars
with high resolution spectroscopy, and provided measurements of
15 different elements (C, N, O, Na, Mg, Al, Si, Ca, Ti, V, Cr, Mn, Fe, Ni, Y)
as well as high precision radial velocities.
We provide all measurements relevant to the present discussion in Table~\ref{tab:t2}.

% ----------------
% Horizontal table
% ----------------
% \begin{table*}[htpb]
%   \centering
%   \caption{Astrometric and spectroscopic measurements}
%   \label{tab:t1}
%   \begin{tabular}{cccccccccc}
% \hline\hline
% Name & Sp Type & $T_\mathrm{eff}$ & $\log{g}$ & $v\sin{i}$ & $[\elem{Fe}/\elem{H}]$ & $v_r$ &
% $\varpi$ & $\mu_\alpha^*$ & $\mu_\delta$ \\
% % units
% \hline
% HD 240429 & G2 & 5878 & 4.43 & 1.1 & 0.01 & $-21.2$  & $9.35 \pm 0.24$ & $89.25 \pm 0.66$ & $-29.68 \pm 0.54$\\
% HD 240430 & G0 & 5803 & 4.33 & 2.5 & 0.20 & $-21.2$  & $9.41 \pm 0.25$ & $89.41 \pm 0.69$ & $-30.12 \pm 0.52$\\
% \hline\hline
% \end{tabular}
% \end{table*}

\begin{table*}[htpb]
  \centering
  \caption{Astrometric and spectroscopic measurements}
  \label{tab:t2}
  \begin{tabular}{ccc}
\hline\hline
Name & HD 240429 & HD 240430 \\
\hline
Sp Type & G2 & G0 \\
$T_\mathrm{eff}$ & 5878 & 5803 \\
$\log{g}$ & 4.43 & 4.33 \\
$v\sin{i}$ & 1.1 & 2.5 \\
$[\elem{Fe}/\elem{H}]$ & 0.01 & 0.20 \\
$v_r$ & $-21.2$ & $-21.2$ \\
$\varpi$ & $9.35 \pm 0.24$ & $9.41 \pm 0.25$ \\
$\mu_\alpha^*$ & $89.25 \pm 0.66$ & $89.41 \pm 0.69$ \\ 
$\mu_\delta$  & $-29.68 \pm 0.54$ & $-30.12 \pm 0.52$\\
\hline\hline
\end{tabular}
\end{table*}

We first confirm that the two stars are indeed co-moving with very similar
velocities using the radial velocities.
\todo{APW: discuss kinematics, posterior of separation/dV}

\todo{Figure: posterior distributions of separation and delta V}

Given their proximity in space and kinematics,
it is generally accepted to assume that the two stars were born coeval
perhaps in a wide binary from the same birth cloud.
Thus, we expect the two stars to have identical metallicities and abundance patterns,
\todo{except for thoese elements like Li...}
Surprisingly, one of the stars, HD 240430 is significantly more metal
rich than the other by 0.2~dex ($\approx 60\%$; Figure~\ref{fig:abundances}).
What is more puzzling is that the abundances of this star
shows selective depletion in C, N, O, Na, and Mn.

\begin{figure}[htpb]
  \centering
  \includegraphics[width=0.9\linewidth]{abundances.pdf}
  \caption{Abundances of the co-moving pair, HD 240429 and HD 240430.}
  \label{fig:abundances}
\end{figure}

\begin{figure}[htpb]
  \centering
  \includegraphics[width=0.9\linewidth]{relabun_tc.pdf}
  \caption{Abundance of HD 240430 ($[\mathrm{Fe}/\mathrm{H}] = 0.20$)
    relative to HD 240429 ($[\mathrm{Fe}/\mathrm{H}] = 0.01$)
    as a function of condensation temperature of elements (\citealt{2003ApJ...591.1220L}).
  }
  \label{fig:relabun_tc}
\end{figure}

\section{Discussion}
\label{sec:discussion}

We discuss the possible origins of this pair.

% chance pair
Given that their metallicities and abundance patterns are significantly
different, one may simply conclude that the two stars are not related (coeval)
but they merely happen to be co-moving at such small separation ($\approx 0.6$~pc)
by chance.
The two stars are in the Galactic disk, and assuming certain velocity ellipsoid
at the stars' location, one may compute the probability that a star moving at
the mean velocity of the two stars would have a companion within $\Delta v = XX$~km/s.
...

% binary-single and binary-binary exchange
Another scenario that two unrelated stars may end up in a wide binary system
is multi-body gravitational scattering.
We consider the exchange probabilities in binary-single and binary-binary interactions.
Given their low binding energy compared to incoming velocity of field stars,
this is low.

% planet-or-the-like engulfment
Yet another possibility is that the more metal-rich of the two stars
was recently? bombarded by something that selectively removed the depleted elements but not the others.

Present the relative abundance figure here

\bibliography{ref}



\end{document}
