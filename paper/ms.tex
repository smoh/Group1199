\documentclass[12pt,letterpaper,margin=1in]{article}
% \documentclass[twocolumn]{aastex61}

% to-do list
% ----------
% - cite gaia dr paper

% style notes
% -----------
% - This file generates by Makefile; don't be typing ``pdflatex'' or some bullshit.
% - Line break between sentences to make the git diffs readable.
% - Use \, as a multiply operator.
% - Reserve () for function arguments; use [] or {} for outer shit.
% - Use \sectionname not Section, \figname not Figure, \documentname not Article or Paper or paper.
% - Use "comoving" instead of "co-moving".
% - Use "phase space" not "phase-space", "phase-space coordinates" not "phase space coordinates".
% - Write elements as \elem{X}.

% packages
\usepackage[letterpaper,margin=1in]{geometry}

\usepackage{setspace}
\onehalfspacing
\usepackage{microtype}  % ALWAYS!
\usepackage{amsmath,amssymb}
\usepackage[sort&compress,super]{natbib}
\usepackage{graphicx}
\graphicspath{{figures/}}
\usepackage{aas_macros}
\usepackage{hyperref}
\hypersetup{backref,breaklinks,colorlinks,citecolor=blue}
\usepackage{mathptmx}

% per section figure counting
\usepackage{chngcntr}
\counterwithin{figure}{section}

%\setcitestyle{super}
\citestyle{nature}
\bibliographystyle{naturemag}

% define macros for text
\newcommand{\project}[1]{\textsl{#1}}
\newcommand{\acronym}[1]{{\small{#1}}}
\newcommand{\gaia}{\project{Gaia}}
\newcommand{\rave}{\project{\acronym{RAVE}}}
\newcommand{\apogee}{\project{\acronym{APOGEE}}}
\newcommand{\tmass}{\project{\acronym{2MASS}}}
\newcommand{\documentname}{Article}
\newcommand{\sectionname}{Section}
\newcommand{\figname}{Figure}
\newcommand{\eqname}{Equation}
\newcommand{\dr}{\acronym{DR1}}
\newcommand{\tgas}{\acronym{TGAS}}
\newcommand{\etal}{\textit{et al}.}
\newcommand*\elem[1]{\ensuremath{\mathrm{#1}}}
\newcommand*\elemH[1]{\ensuremath{[\mathrm{#1}/\elem{H}]}}
\newcommand*\teff{\ensuremath{T_\mathrm{eff}}}
\newcommand*\logg{\ensuremath{\log{g}}}
\newcommand*{\feh}{\ensuremath{\elemH{Fe}}}
\newcommand{\sunanalog}{\text{Krios}}
\newcommand{\bizarreone}{\text{Kronos}}
\newcommand{\Tcondens}{\ensuremath{T_C}}
\newcommand{\mearth}{\ensuremath{M_\oplus}}
\newcommand{\mjupiter}{\ensuremath{M_\mathrm{Jup}}}

% define macros for math
\newcommand{\given}{\,|\,}
\newcommand{\normal}{{\mathcal{N}}}
\newcommand{\dd}{\mathrm{d}}
\newcommand{\transp}[1]{{#1}^{\!\mathsf{T}}}
\newcommand{\inv}[1]{{#1}^{-1}}
\newcommand{\bs}[1]{\boldsymbol{#1}}
\newcommand{\vperp}{\bs{v}^\perp}
\newcommand{\propm}{\bs{\mu}}
\newcommand{\mat}[1]{\mathbf{#1}}
\renewcommand{\vec}[1]{\bs{#1}}
\newcommand{\kms}{\ensuremath{\rm km~s^{-1}}}
\newcommand{\msun}{\ensuremath{{\mathrm M}_\odot}}
\newcommand{\pc}{{\rm pc}}
\newcommand{\data}{\mathrm{data}}
\newcommand{\snr}{[S/N]_\varpi}
\newcommand{\eye}{\mathbb{I}}
\newcommand{\absdvtan}{\ensuremath{|\Delta\vec v_\mathrm{t}|}}
\newcommand{\estimates}{\ensuremath{\{\hat{\varpi_i},\hat{\mu_{\alpha,i}},\hat{\mu_{\delta,i}},\hat{v_{r,i}}\}}}

\newcommand{\todo}[1]{{TODO: #1}}

\renewcommand\tablename{Table}

\begin{document}\sloppy\sloppypar\raggedbottom\frenchspacing % trust me

\title{
  Kronos \& Krios:
  Evidence for accretion of a massive, rocky planetary system
  in a comoving pair of G dwarfs
}
\maketitle

\author{
  Semyeong Oh$^{1,*}$,\,
  Adrian M. Price-Whelan$^{1}$,\,
  John M. Brewer$^{2}$,\,
  David W. Hogg$^{3,4,5,6}$,\,
  David N. Spergel$^{1,3}$,\,
  Justin Myles$^{2}$
}

% affiliations
{\small\noindent
  $^*$To whom correspondence should be addressed: \texttt{semyeong@astro.princeton.edu} \\
  $^1$Department of Astrophysical Sciences, Princeton University, 4 Ivy Lane, Princeton, NJ 08544, USA \\
  $^2$Department of Astronomy, Yale University, 260 Whitney Ave, New Haven, CT 06511, USA \\
  $^3$Center for Computational Astrophysics, Flatiron Institute, 162 Fifth Ave, New York, NY 10010, USA \\
  $^4$Center for Data Science, New York University, 60 Fifth Ave, New York, NY 10011, USA \\
  $^5$Center for Cosmology and Particle Physics, Department of Physics, New York University, 726 Broadway, New York, NY 10003, USA \\
  $^6$Max-Planck-Institut f\"ur Astronomie, K\"onigstuhl 17, D-69117 Heidelberg \\
}

%% \author{Keith A. Hawkins}
%% \affil{Department of Astronomy, Columbia University, 550 W 120th St, New York, NY 10027, USA}

%% \author{Nathan W. C. Leigh}
%% \affil{Department of Astrophysics, American Museum of Natural History, Central Park West and 79th St, New York, NY 10024, USA}


% ------------------------
% nature summary paragraph
% ------------------------
% - a fully referenced paragraph, ideally of about 200 words, but certainly no
%   more than 300 words, aimed at readers in other disciplines.
% - This paragraph starts with a 2-3 sentence basic introduction to the field;
%   followed by a one-sentence statement of the main conclusions starting 'Here
%   we show' or equivalent phrase; and finally, 2-3 sentences putting the main
%   findings into general context so it is clear how the results described in
%   the paper have moved the field forwards.
% ------------------------
\section{Summary}
{\bf \noindent
  Stars that are born together are expected to have same chemical compositions initially.
  Detailed chemical abundance studies of stars in wide binaries indeed reveal nearly identical
  chemistry in most cases\citep{Gratton:2001aa,Desidera:2004aa} as well as stars in open clusters\cite{Bovy:2016aa}.
  However, formation and evolution of different planetary systems around each star in a wide binary
  may eventually lead to a difference in the compositions of the component stars' radiative and
  convective layers\cite{Pinsonneault:2001aa,Chambers:2010aa}.
  Wide binary systems of two twin-like stars are particularly useful
  as they are not subject to confusion with Galactic chemical evolution.
  A handful of wide binaries that show small differences in their surface
  chemical abundances have been reported so far
  \cite{Mack:2014aa,Mack:2016aa,Saffe:2015aa,Teske:2013aa,Teske:2015aa,Teske:2016aa,Teske:2016ab,Biazzo:2015aa,Ramirez:2015aa}.
  However, a convincing example is yet to be found.
  Here we report a comoving pair of bright sun-like stars, HD~240430 and HD~240429,
  with a significant difference ($\approx 0.2$~dex) in their chemical
  abundances that strongly suggests accretion of rocky planetary system in one
  of the stars within the last couple of billion years.
  The more metal-rich of the two, HD~240430, shows an enhancement of refractory
  ($\Tcondens>1200$~K) elements by $\approx 0.2$~dex but not as large
  an enhancement of (moderately) volatile elements ($\Tcondens<1200$~K; \elem{C},
  \elem{N}, \elem{O}, \elem{Na}, and \elem{Mn}).
  The two stars also show high Li abundance given their age of $\sim 4$ Gyr
  inferred from their stellar parameters.
  A prominent trend of element abundance enhancement with condensation temperature
  indicates the accreted material is rocky.
  We therefore suggest that the star HD~240430, ``Kronos'', recently accreted
  $\approx 20$~\mearth\ of rocky material after birth selectively enhancing the
  refractory elements in its surface and convective envelope.
  The other star HD~240429, ``Krios'', may also have had a similar accretion
  event given its high surface \elem{Li} abundance but of much smaller amount
  ($\approx 4$~\mearth).
  This pair of stars provides evidence that dynamical accretion of rocky planets
  happen during the lifetime of sun like stars.
}

\section{Main Text}

\begin{figure}[htpb]
  \centering
  \includegraphics[width=0.9\linewidth]{abundances.pdf}
  \caption{Abundances of the comoving pair, \sunanalog\ and \bizarreone,
    normalized to \elem{H} (top) and \elem{Fe} (bottom).
    Lines are drawn for each star only to guide the eye.
    \bizarreone\ is enhanced in \elem{Fe} by $\approx 0.2$~dex relative to
    \sunanalog\ along with \elem{Mg}, \elem{Al}, \elem{Si}, \elem{Ca},
    \elem{Ti}, \elem{V}, \elem{Cr}, \elem{Ni}, \elem{Y} yet not in \elem{C},
    \elem{N}, \elem{O}, \elem{Na}, and \elem{Mn}.
    \label{fig:abundances}
  }
\end{figure}

\begin{table*}[htpb]
  \caption{Astrometric and spectroscopic measurements.}
  \label{tab:t1}
  \centering
  \begin{tabular}{ccc}
\hline\hline
Name & HD 240429 & HD 240430 \\
\hline
Sp Type                                   & G2                & G0                \\
$T_\mathrm{eff}$                          & 5878              & 5803              \\
$\log{g}$                                 & 4.43              & 4.33              \\
$v\sin{i}$                                & 1.1               & 2.5               \\
$[\elem{Fe}/\elem{H}]$                    & 0.01              & 0.20              \\
$v_r$                                     & $-21.2$           & $-21.2$           \\
$\varpi$ \footnotemark[1]                 & $9.35 \pm 0.24$   & $9.41 \pm 0.25$   \\
$\mu_\alpha^*$ \footnotemark[1]           & $89.25 \pm 0.66$  & $89.41 \pm 0.69$  \\
$\mu_\delta$ \footnotemark[1]             & $-29.68 \pm 0.54$ & $-30.12 \pm 0.52$ \\
\hline\hline 
\end{tabular}
\end{table*}


\sunanalog\ and \bizarreone\ were identified as a candidate comoving star pair
in our recent search for comoving stars using the proper motions and parallaxes
from the {\it Tycho-Gaia Astrometric Solution} catalog (hereafter \tgas), a
component of the first data release of the astrometric space mission \gaia\
(the astrometric measurements are listed in \tablename~\ref{tab:t1}).
The pair has also been previously recognized as a visual double star system
in Washington Double Star catalog\cite{2001AJ....122.3466M}.
The two stars have spectral types G0 and G2 both similar to the Sun (G2).
In a separate effort to study detailed chemical abundances of potential
planet-hosting stars, high resolution, high signal-to-noise ratio spectra of
both stars were obtained using the HIRES spectrograph on the Keck-I
telescope\cite{2016ApJS..225...32B}. 
The spectral resolution is $R=\lambda/\Delta\lambda\approx 70000$ and the
wavelength coverage is $5164$--$7799$~\AA.
A typical signal-to-noise ratio in the spectral continuum is $>200$~per pixel.
The resulting measurements include elemental abundances for 15 chemical species
(C, N, O, Na, Mg, Al, Si, Ca, Ti, V, Cr, Mn, Fe, Ni, Y) as well as stellar
parameters and high precision ($\sigma\approx0.2$~\kms) radial velocities.

The projected separation between the pair is 1.9 arcmin\ ($\approx 0.01$~pc),
and the 3D separation is $\approx 0.6$~pc.
Although selected based only on their astrometry, the two stars
have identical radial velocities within uncertainties (Table~\ref{tab:t1}),
confirming that they are truly comoving.
As the separation is smaller than the Jacobi radius (1.2~pc) in the solar
neighborhood for a 2~\msun\ binary system\cite{Jiang:2010aa}, \bizarreone\ and
\sunanalog\ are likely a bound wide binary system that formed coevally.

Aside from the kinematics, their stellar parameters also indicates that they are coeval.
We use the distances (inferred from \gaia\ parallaxes), $V$-band magnitudes,
and $B-V$ colors to obtain bolometric luminosities of the two stars\cite{2003AJ....126..778V}.
We then combine the luminosities with $T_\mathrm{eff}$, \elemH{Fe}, \elemH{Si}
in order interpolate the mass, age, radius, and $\log(g)$ of each star
in a grid of Yale-Yonsei model isochrones\cite{2013ApJ...776...87S}.
The best-fit isochrone ages of \sunanalog\ and \bizarreone\ are
$4.00_{-1.56}^{+1.51}$~Gyr and $4.28_{-1.03}^{+1.11}$~Gyr, respectively,
consistent with them being coeval.

However, one of the stars, \bizarreone\ is significantly more metal
rich than the other (by 0.2~dex $\approx 60\%$; \figname~\ref{fig:abundances}).
Moreover, not all elements are equally enhanced:
the abundances of \bizarreone\ show selective depletion in
\elem{C}, \elem{N}, \elem{O}, \elem{Na}, and \elem{Mn}
relative to \elem{Fe}.
%
The validity of the measured abundance differences is further demonstrated
in \figname~\ref{fig:spec1} and \ref{fig:spec2} where we show
segments of the spectra and models of the two stars
used to measure their abundances.
As expected from their reported metallicity difference ($\Delta\feh \approx 0.2$),
the ratio of data and model between the two stars show significant
residuals of almost all metal line features, largely dominated by \elem{Fe}.
However, for lines of elements that are not as enhanced in \bizarreone\,
the residuals are much smaller in amplitude.

The observed difference in metallicity and abundances are genuinely rare and
unexpected.
None of the other four twin-like ($\Delta T_\mathrm{eff} \lesssim 100$~K) wide
binary pairs examined using spectra of comparable quality and the same
consistent measurement pipeline shows discrepancies in abundances between the
stars at this level\cite{2016ApJS..225...32B} (see Figure~\ref{fig:deltaXH}).
The maximum metallicity difference observed in the other pairs is $\Delta\feh = 0.03$~dex
with $1-\sigma$ statistical uncertainties in \feh\ of $0.01$~dex.
This is consistent with similar studies of metallicity and abundance differences
in wide binaries\cite{Gratton:2001aa,Desidera:2004aa}.
The difference of $\approx 0.2$~dex is also the largest among a handful of
wide binaries that do show metallicity and abundance differences with varying
significance\cite{Mack:2014aa,Mack:2016aa,Saffe:2015aa,
  Teske:2013aa,Teske:2015aa,Teske:2016aa,Teske:2016ab,Biazzo:2015aa,Ramirez:2015aa}
(see Figure~\ref{fig:relabun_tcrank}).

%As shown in \figname~\ref{fig:deltaXH},
%the differences in other pairs for all elements except \elem{N} and \elem{O},
%which are also the most uncertain (\tablename~\ref{tab:kk}),
%are less than $0.05$~dex, making \bizarreone-\sunanalog\ pair a significant outlier.

Both stars have high surface \elem{Li} abundances considering
their ages of $\sim 4$~Gyr.
The surface lithium abundance in a sun-like star decreases with its age due to
mixing induced by convection or rotation, which brings the lithium into the
interior ($T>2.5 \times 10^{6}$~K) where it will be destroyed by proton capture
burning.
The \elem{Li} doublet at $6707.6$~\AA, which is usually too weak to be
detectable at this age even for high signal-to-noise spectra, is clearly
visible in the spectra of both stars (see \figname~\ref{fig:spec2}).
The measured absolute abundances for \bizarreone\ and \sunanalog\ are
$2.75$~dex and $2.25$~dex, respectively (Myles \etal, in prep),
resulting in the largest abundance difference ($\approx 0.5$~dex) seen
among all measured elements.


\begin{figure}[htpb]
  \centering
  \includegraphics[width=0.95\linewidth]{toycalc.png}
  \caption{
    Comparing the observed abundance difference vs. \Tcondens\ rank
    to the expected change in solar surface abundance after adding $20$~\mearth\ of
    material with bulk Earth composition (\cite{mcdonough2001composition}).
    The assumed mass fraction in the convective zone is $0.03$.
    All metals are ranked by their \Tcondens\ for solar composition gas,
    and the condensation temperature may be read from the gray line and right y-axis,
    same as in \figname~\ref{fig:relabun_tcrank}.
    Locations of the elements measured for \bizarreone-\sunanalog\ pair are
    indicated by a vertical line and its symbol.
    The close match with the observed abundance difference in \bizarreone-\sunanalog\ pair
    suggests that the abundance difference may be due to accretion of
    $20$~\mearth\ of rocky planetary material.
    The element \elem{Li} is off the plot and indicated with a red arrow
    (see text for details).
  }
  \label{fig:toycalc}
\end{figure}

In Figure~\ref{fig:relabun_tcrank}, we show the abundance difference
between \bizarreone\ and \sunanalog\ ordered by the rank of \Tcondens\
of each element.
The equilibrium condensation temperatures for the composition of solar system
are taken from \cite{2003ApJ...591.1220L} (Table~8).
The difference seen in \bizarreone-\sunanalog\ is
compared to HD~20781/2, XO-2N/S, WASP-94A/B in \figname~\ref{fig:relabun_tcrank}.
The metallicity difference of $\approx 0.2$~dex observed in this pair
is larger than the differences seen in any other pairs studied so far.
Strikingly, the five under-enhanced elements in \bizarreone\
relative to \sunanalog\ are the five most volatile in all elements measured.
The difference in \elem{Mn} ($\Tcondens = 1158$~K) and
\elem{Cr} ($\Tcondens = 1296$~K) suggests a break in $\Tcondens \approx 1200$~K.
This $\Tcondens$-dependent trend of $\Delta\elemH{X}$,
combined with the enhanced $\elem{Li}$ abundance ($A(\elem{Li}) = 2.75$),
strongly suggests that accretion of rocky material has occured in \bizarreone.

We interpret the \Tcondens-dependent difference of elemental abundance between
the two stars as evidence for accretion of rocky planetary material in
\bizarreone.
Instabilities may develop in a multi-planet system due to its chaotic nature
which may lead to planet engulfment or ejection by the host star
(\todo{needs references}).
%TODO: kozai?


%%TODO: cautions on condensation temperature
%% 1. it is just a proxy
%% 2. it is **equlibrium** condensation temperature


We estimate how much mass of rocky material is needed to explain an increment of
$\approx 0.2$~dex in refractory elements.
We carry out simple toy calculations of expected $\Delta\elemH{X}$
in a Sun-like star's atmosphere by adding certain mass of bulk Earth composition
under these simplifying assumptions:
\begin{itemize}
  \item The material added is instantly and completely mixed.
  \item The atmospheric composition that we measure is identical throughout
    the star's radiative and convective zone.
  \item The surface abundance of the star has been altered only by the
    accretion event(s).
\end{itemize}
We take the solar abundances, $\elemH{X}$, of element \elem{X}
(\cite{Asplund:2009aa}) which can be converted to mass fraction as
\begin{equation}
  f_{X,\mathrm{photo}} = \frac{10^{\elemH{X}} m_X}{\Sigma_X 10^{\elemH{X}} m_X}
\end{equation}
where $m_X$ is the mass of each element in e.g., atomic mass unit.
Assuming that the added material has a total mass $M_\mathrm{add}$, and the
mass fraction in each element $f_{X,\mathrm{add}}$,
the abundance difference is
\begin{equation}
  \Delta\elemH{X} = \log_{10} \frac{f_{X,\mathrm{photo}}\,f_\mathrm{CZ}\,M_\mathrm{star}
    + f_{X,\mathrm{add}}\,M_\mathrm{add}}
    {f_{X,\mathrm{photo}}\,f_\mathrm{CZ}\,M_\mathrm{star}}
\end{equation}
where $f_\mathrm{CZ}$ is the fraction of the star's mass in the convective envelope.
We take the composition of bulk Earth from a chondritic model of the Earth
(\cite{mcdonough2001composition}).
Similar calculations have been performed by \citet{Chambers:2010aa} and
\citet{Mack:2014aa,Mack:2016aa}.

Figure~\ref{fig:toycalc} shows the expected change of surface abundances of
metals in a Sun-like star after $20~\mearth$ of material with composition of
bulk Earth is added.
A volatility trend that more volatile (low \Tcondens) elements are more
depleted in the Earth relative to CI or other carbonaceous chondrites
has long been known (\cite{mcdonough2001composition}).
This trend is presumed to be closely related to the formation of terrestrial
planets and, in particular to the radial temperature gradient in a
protoplanetary disk.
The trend resulting from adding $20~\mearth$ of bulk Earth
provides an overall good match to the observed $\Delta\elemH{X}$,
suggesting that the refractory-enhanced star, \bizarreone\,
accreted $20~\mearth$ more of rocky planetary material than \sunanalog.

What about \elem{Li}?
The element \elem{Li} is worth special attention in the context of the
accretion scenario.
Because Li is present in either carbonaceous chondrites or bulk Earth with
a concentration of $1-1.5$~ppm in mass (\cite{mcdonough2001composition}),
but is depleted quickly within the first couple of Gyrs on the surface of a
Sun-like star, accretion of either material at later times will significantly
replenish the lithium on the star's surface.
For the present-day Sun, the accretion of $20~\mearth$ of bulk Earth-like
material would result in $\Delta\elemH{Li} \approx 1.6$~dex (which
is indicated as an upward arrow in \figname~\ref{fig:toycalc}).
Incidentally, the star under examination, \bizarreone, has an age (informed by
stellar parameters) very close to the Sun ($4.28_{-1.03}^{+1.11}$~Gyr).
Thus, as long as we believe the Sun's present-day surface \elem{Li} abundance
to be ordinary, the accretion of $20$~\mearth\ by \bizarreone\ is consistent with $1.6$~dex
enhancement in its \elem{Li}.
This is exactly what we find: the \elem{Li} abundance of \bizarreone\ is
$A(\elem{Li}) = 2.75$ (Table~\ref{tab:kk}, \cite{jmlithium})
approximately $1.65$~dex higher than the solar value of $1.1$~dex
(\todo{needs reference}).
Furthermore, this implies that the accretion event should have happened very
recently, or at least within the \elem{Li} depletion time ($\lesssim 1$~Gyr).

What about \sunanalog?
Considering its age of $4.00_{-1.56}^{+1.51}$~Gyr, \sunanalog\ is also enhanced
in \elem{Li} compared to other stars of similar ages by $\approx 1$~dex.
This enhancement puts an upper limit on the accreted mass to be
$\approx 4$~\mearth\ assuming \elem{Li} concentration of $1.1$~ppm
(\cite{mcdonough2001composition}).
Unlike \bizarreone, we cannot know the pre-accretion abundances of
\sunanalog.\footnote{
  Strictly speaking, we can never know the pre-accretion abundances
  for \bizarreone\ either.
  We only have an {\it approximate} idea that this is not far from a solar-twin
  star \sunanalog, and because the deviation from this ``anchor'' star is
  large, it is reasonable to consider accretion by the Sun.
  In reality, \sunanalog\ may as well have had its own accretion history,
  which indeed seems to be the case according to its \elem{Li} abundance.
}
However, we note that the span of abundance difference between highly volatile
elements (\elem{C}, \elem{N}, \elem{O}, \elem{Na}) and refractory elements such
as \elem{Fe} expected from accreting $4$~\mearth\ to the Sun is $\approx
0.05$~dex.
This is comparable to the abundance difference between N, Na and Fe in
\sunanalog.
It is also interesting that \sunanalog\ shows a deficit in the same volatile
and moderately volatile elements (\elem{C}, \elem{N}, \elem{O}, \elem{Na}, and
\elem{Mn}) relative to Fe as in \bizarreone.
Thus, we conclude that \sunanalog\ is also likely to have had a similar
accretion event, but the amount of accretion was at least 5 time
smaller than \bizarreone.

%TODO: augment this paragraph
% - more geochemical study references
We stress that while the calculation carried out is useful in
an order-of-magnitude sense, further investigation of each of the simplifying
assumptions made is warranted.
In addition, the composition of bulk Earth has some uncertainties.
For example, the reported bulk Earth concentration of the siderphile element
\elem{Mn}, varies from $800$ to $1700$~ppm (\cite{1998psc..book.....L}).
While the latter value from \cite{mcdonough2001composition} has been used in
our calculation, the former value would bring the observed $\Delta\elemH{Mn}$
to an even closer agreement.
Given these limitations, the level of agreement for $\Delta\elemH{X}$ {\it and}
\elem{Li} for \bizarreone\ is remarkable.

Given that the two stars must have engulfed planets,
it is likely that they have not destroyed all of their planets.
Any surviving planet around the stars must be giants.
Fortunately, discovery of giant eccentric planets using astrometric jitter is in the sweet spot
for Gaia.
The two stars have not been included in any publicly released data from planet
search programs.
If both stars have accreted planetary material, it would be very interesting to
search for the existence and architectures of the planetary systems left
behind.


\bibliography{ref}

\section{Acknowledgements}
We thank Megan Bedell, Andy Casey, Keith Hawkins, Nathan Leigh, and Josh Winn
for feedback.
The Flatiron Institute is supported by the Simons Foundation.
% Gaia
This work has made use of data from the European Space Agency (ESA) mission
{\it Gaia} (\url{http://www.cosmos.esa.int/gaia}), processed by the {\it Gaia}
Data Processing and Analysis Consortium (DPAC,
\url{http://www.cosmos.esa.int/web/gaia/dpac/consortium}). Funding for the DPAC
has been provided by national institutions, in particular the institutions
participating in the {\it Gaia} Multilateral Agreement.
% software
%\software{
%  %The data and code used in this project is available from
%  %\url{https://github.com/smoh/KronosKrios} under the MIT open-source
%  %software license.
%  This research utilized:
%  \texttt{Astropy} (\citealt{Astropy-Collaboration:2013}),
%  \texttt{emcee} (\citealt{2013PASP..125..306F}),
%  \texttt{IPython} (\citealt{Perez:2007}),
%  \texttt{matplotlib} (\citealt{Hunter:2007}),
%  \texttt{numpy} (\citealt{Van-der-Walt:2011}),
%  and \texttt{pandas} (\citealt{pandas}).}


\section{Extended Data}

\begin{table*}[htpb]
  \caption{Astrometric and spectroscopic measurements.}
  \label{tab:t2}
  \centering
  \begin{tabular}{ccc}
\hline\hline
Name & HD 240429 & HD 240430 \\
\hline
Sp Type                                   & G2                & G0                \\
$T_\mathrm{eff}$                          & 5878              & 5803              \\
$\log{g}$                                 & 4.43              & 4.33              \\
$v\sin{i}$                                & 1.1               & 2.5               \\
$[\elem{Fe}/\elem{H}]$                    & 0.01              & 0.20              \\
$v_r$                                     & $-21.2$           & $-21.2$           \\
$\varpi$ \footnotemark[1]                 & $9.35 \pm 0.24$   & $9.41 \pm 0.25$   \\
$\mu_\alpha^*$ \footnotemark[1]           & $89.25 \pm 0.66$  & $89.41 \pm 0.69$  \\
$\mu_\delta$ \footnotemark[1]             & $-29.68 \pm 0.54$ & $-30.12 \pm 0.52$ \\
\hline
\multicolumn{3}{c}{$T_c < 1200$~K} \\
\hline
$A(\elem{Li})$ \footnotemark[2]           & $2.25$            & $2.75$            \\
$\elemH{C}$                               & $0.00$            & $0.09$            \\
$\elemH{N}$                               & $-0.06$           & $-0.01$           \\
$\elemH{O}$                               & $0.01$            & $0.09$            \\
$\elemH{Na}$                              & $-0.06$           & $-0.04$           \\
$\elemH{Mn}$                              & $-0.03$           & $0.00$            \\
\hline
\multicolumn{3}{c}{$T_c > 1200$~K} \\
\hline
$\elemH{Mg}$                              & $0.01$            & $0.19$            \\
$\elemH{Al}$                              & $0.01$            & $0.21$            \\
$\elemH{Si}$                              & $0.00$            & $0.16$            \\
$\elemH{Ca}$                              & $0.02$            & $0.23$            \\
$\elemH{Ti}$                              & $0.02$            & $0.20$            \\
$\elemH{V}$                               & $0.02$            & $0.20$            \\
$\elemH{Cr}$                              & $0.01$            & $0.17$            \\
$\elemH{Fe}$                              & $0.01$            & $0.20$            \\
$\elemH{Ni}$                              & $-0.01$           & $0.21$            \\
$\elemH{Y}$                               & $0.04$            & $0.26$            \\
\hline\hline
\end{tabular}
%
%\begin{tablenotes}
%  \item All spectroscopic measurements except $\elem{Li}$ are from \citealt{2016ApJS..225...32B}.
%  \footnotetext[1]{From \tgas. }
%  \footnotetext[2]{Absolute abundance from \todo{CITE}.}
%\end{tablenotes}
\end{table*}

\begin{figure}[htpb]
  \centering
  \includegraphics[width=0.95\linewidth]{deltaXH_elem_violins.pdf}
  \caption{Abundance difference in this pair and other twin-like
    ($\Delta T_\mathrm{eff}\lesssim 100$~K) wide binaries in
    \citealt{2016ApJS..225...32B}.
    The differences in other pairs are small ($<0.05$~dex)
    for all elements except \elem{N} and \elem{O} which are the most
    uncertain, making the difference of $\approx 0.2$~dex seen in
    \bizarreone-\sunanalog\ rare.
    Additionally, we show the distribution of abundance differences
    between field stars with similar metallicity difference
    ($\Delta[\elem{Fe}/\elem{H}] \approx 0.2$)
    as violins with medians indicated by black line segments.
    These are random pairings of single stars in
    in \citealt{2016ApJS..225...32B} at two metallicity bins,
    $-0.025 < \feh < 0.025$ (160 stars) and $0.175 > \feh > 0.225$ (137 stars),
    similar to \bizarreone\ and \sunanalog.
    The difference is always taken to be
    $\mathrm{higher}\,\feh - \mathrm{lower}\,\feh$.
    Thus, the narrower range of in $\Delta\feh$ is by construction.
    Random pairings of disk stars with similar $\Delta\feh$ usually show
    similar enhancement in all other elements
    unlike the pattern seen in \bizarreone-\sunanalog\ pair.
  }
  \label{fig:deltaXH}
\end{figure}



\begin{figure}[htpb]
  \centering
  \includegraphics[width=0.95\linewidth]{TcRank_deltaXH_concise.pdf}
  \caption{Abundance differences of the \bizarreone-\sunanalog\ pair
    ranked by the condensation temperature of elements for solar composition gas
    from \citet{2003ApJ...591.1220L}.
    The condensation temperature may be read from the gray line and right y-axis.
    We show three wide binary systems selected from the literature:
    HD~20782/1 (\citet{Mack:2014aa}, $\feh\approx0$),
    XO-2N/S\cite{Biazzo:2015aa} ($\feh\approx0.35$)),
    and WASP-94AB\cite{Teske:2016aa} ($\feh\approx0.3$).
    Locations of elements with at least one measurement from any study
    are indicated by a vertical line and its symbol.
    Note that often multiple values are reported for one element corresponding
    to different ionization states in equivalent width analyses.
    No other pair studied so far were shown to have such large difference
    in metallicity or sharp contrast between (moderately) volatile and
    refractory elements as \bizarreone-\sunanalog.
  }
  \label{fig:relabun_tcrank}
\end{figure}



\begin{figure}[htpb]
  \centering
  \includegraphics[width=0.95\linewidth]{spec1.pdf}
  \caption{Selective segments of the spectra of \sunanalog\ and \bizarreone.
    Alternating sets of two rows show
    the continuum-normalized data and model in the upper panel,
    and the ratio (\bizarreone/\sunanalog) of data (gray) and model (black)
    in the lower panel.
  }
  \label{fig:spec1}
\end{figure}

\begin{figure}[htpb]
  \centering
  \includegraphics[width=0.95\linewidth]{spec2.pdf}
  \caption{Same as \figname~\ref{fig:spec1}
    but for smaller portions of spectra at longer wavelengths that are
    not dominated by \elem{Fe}.
    We mark elements that give rise to strong absorption lines.
    The feature in the third column of the second row is a strong \elem{Li}
    line that was not modeled by \citealt{2016ApJS..225...32B}; this line was
    studied in a separate work (\citealt{jmlithium}).
    Note that the lines of \elem{Na} and \elem{O}, which are under-enhanced
    in \bizarreone\ relative to \elem{Fe} or other refractory elements,
    show weaker residuals.
  }
  \label{fig:spec2}
\end{figure}

%TODO: plot styles...
\begin{figure}[htbp]
  \begin{center}
    \includegraphics[width=\linewidth]{orbits.pdf}
  \end{center}
  \caption{Left panel: Galactic orbits computed for \sunanalog\ (black) and the
    Sun (grey).
    For \sunanalog, the initial conditions are set to the median of the
    posterior samples over the phase-space coordinates.
    The orbits are computed by integrating backwards from the present-day
    positions for $2.5$~Gyr with a time step of $0.5$~Myr using the Leapfrog
    integration scheme implemented in \project{Gala} (\citealt{gala}).
    Right panel: distribution of maximum $z$-heights for orbits computed from
    all posterior samples.
  }
  \label{fig:orbit}
\end{figure}


\section*{Supplementary Information}

\subsection*{Proximity in phase-space coordinates}

\begin{figure}[htbp]
  \begin{center}
    \includegraphics[width=\linewidth]{dx_dv_posterior.pdf}
  \end{center}
  \caption{%
    Differences in posterior samples over Galactocentric phase-space coordinates
    for the two stars \sunanalog\ and \bizarreone.
    \label{fig:dxdv}}
\end{figure}

Combining the precise radial velocities obtained from spectra with the \tgas\
astrometry, we can compare differences between the inferred 6D phase-space
coordinates of the two stars.
We start by generating posterior samples over the Heliocentric distance, $r$,
tangential velocities, $(v_\alpha, v_\delta)$, and radial velocity, $v_r$,
given the observed parallax, $\hat\pi$, and proper motions,
$(\hat\mu_{\alpha^*}, \hat\mu_\delta)$, and radial velocity, $\hat v_r$.
We assume the noise is Gaussian, and the radial velocity measurement is
uncorrelated with the astrometric measurements.
If we define
\begin{eqnarray}
  \vec{\hat y} &=&
      \transp{\left(
        \begin{array}{c@{\hspace{1em}} c@{\hspace{1em}} c@{\hspace{1em}} c}
          \hat\pi &
          \hat\mu_{\alpha^*} &
          \hat\mu_\delta &
          \hat v_r
        \end{array}
      \right)}\\
  \vec{y} &=&
      \transp{\left(
        \begin{array}{c@{\hspace{1em}} c@{\hspace{1em}} c@{\hspace{1em}} c}
          r^{-1} &
          r^{-1}\,v_\alpha &
          r^{-1}\,v_\delta &
          v_r
        \end{array}
      \right)}
\end{eqnarray}
then the likelihood is
\begin{equation}
  \vec{\hat y} \sim \mathcal{N}(\vec{y}, \mat{C})
\end{equation} where $\mat{C}$ is the covariance matrix.
We adopt a uniform space density prior for the distance and an isotropic
Gaussian for any velocity component, $v$, with a dispersion $\sigma_v=25~\kms$
\begin{eqnarray}
p(r) &=&
  \begin{cases}
    \frac{3}{r_{\rm lim}^3} \, r^2 & \text{if } 0 < r < r_{\rm lim}\\
    0              & \text{otherwise}
  \end{cases}\\
p(v) &=& \frac{1}{\sqrt{2\pi}\,\sigma_v} \,
  \exp\left[-\frac{1}{2} \, \frac{v^2}{\sigma_v^2} \right] \quad .
\end{eqnarray}
%
For each of the two stars, we use \project{emcee}
(\citealt{2013PASP..125..306F}) to generate posterior samples in $(r, v_\alpha,
v_\delta, v_r)$ by running 64 walkers for 4608 steps and discarding the first
512 steps as the burn-in period.
For each sample, we convert the heliocentric phase-space coordinates into
Galactocentric coordinates assuming that the Sun's position and velocity are
$\vec x_\odot = (-8.3,\,0,\,0)~{\rm kpc}$ and $\vec v_\odot =
(-11.1,\,244,\,7.25)~\kms$ \citep[e.g.,][]{Schonrich:2010, Schonrich:2012}.

\figurename~\ref{fig:dxdv} shows differences in posterior samples converted to
Galactocentric phase-space coordinates for the two stars.
The differences in positions and velocities are consistent with zero.
\todo{should we instead show $|\Delta\boldsymbol{x}|$ and
$|\Delta\boldsymbol{v}|$ computed from the 6D posterior samples?}

\subsection*{}

The surface lithium abundance in a sun-like star decreases with its age due to
mixing induced by convection or rotation, which brings the lithium into the
interior ($T>2.5 \times 10^{6}$~K) where it will be destroyed by proton capture
burning.
Thus, surface lithium abundance is an indicator of stellar ages.
The absolute $\elem{Li}$ abundance of $2.25$~dex for \sunanalog\ implies an age
of $\lesssim 1$~Gyr according to the theoretical models tuned to explain the
solar lithium abundance and rotational profile\cite{2005Sci...309.2189C}.
The lithium abundance of \bizarreone\, $2.74$~dex, shows the largest difference
among all measured elements.
This translates to a $\sim 500$~Myr difference in age.
Given the overall higher metal abundances and the peculiar abundance patterns
in \bizarreone, it is unclear, however, whether this higher $\elem{Li}$
abundance means a younger age or something else.
For example, the presence of $\elem{Li}$-rich red giant stars has been
attributed to the engulfment of substellar companions such as gas giant planets
or brown dwarfs which may replenish $\elem{Li}$\cite{Casey:2016aa}.

In fact, the surface lithium abundance is the only indication that points to
younger ages.
If the two stars are indeed only several hundred Myrs old at most,
they are expected to be part of a larger comoving group of stars.
However, as we mention above, there is no evidence in our search of comoving pairs
using \tgas\ that the two stars belong to a larger group of young stars.
Very young stars often show signs of activity such as
X-ray emission from magnetic activity, emission lines, or infrared excess due to
circumstellar disks (\todo{needs refernces}).
We have compiled GALEX, Tycho-2, 2MASS, and WISE photometry for these stars,
and found no evidence for indications of activity in their spectral energy
distributions.
The low $v\sin(i)$ values (\tablename~\ref{tab:kk}) also argue against very
young ages that would be inferred from the surface lithium abundances.
Finally, we computed the Galactic orbit of the pair using the median of the
posterior sample over the phase space coordinates of \sunanalog, in a Milky
Way-like gravitational potential (similar to \texttt{MWPotential2014} from
\cite{Bovy:2015}) using \project{Gala} (\cite{gala}).
The pair's fiducial orbit has a vertical action larger than the Sun, favoring
an older age (\todo{needs references for vertical action as age indicator}).
We therefore conclude that the two stars are most likely coeval, $\sim 4$~Gyr
old main sequence stars, and that their unusually high \elem{Li} abundance
requires an alternative explanation.

\subsection{Chemical Inhomogeneity in Star Formation}
\label{sub:chemical_inhomogeneity_in_star_formation}

Assuming that the two stars were born together, one possibility to
explain the chemical difference is that there was chemical inhomogeneity within
the birth cloud.
However, there is already ample evidence against such a scenario.
First, none of the other seven similar wide binaries examined in
\citealt{2016ApJS..225...32B} show the same level of differences in abundances.
Though there is generally a larger spread in $\elem{C}$, $\elem{N}$ and
$\elem{O}$, and some pairs show a difference in particular elements as large as
$\approx 0.15$~dex.
The median and maximum \feh\ difference between component stars in the other
seven pairs is $0.02$~dex and $0.09$~dex, respectively.
The differences are even smaller (maximum $\Delta\feh = 0.03$~dex) if we
compare only twin-like ($\Delta T_\mathrm{eff} \lesssim 100$~K) pairs
(\figname~\ref{fig:deltaXH}).
This is consistent with the findings of \citealt{Desidera:2004aa}, who examined
23 wide binaries of late F to K dwarfs, and found most pairs show a difference
in \feh\ less than $0.02$~dex and none larger than $0.07$~dex.
Similarly, \citealt{Gratton:2001aa} found that four out of six equal mass
binaries have the same chemical composition with $\gtrsim 0.01$~dex
uncertainties.
Thus, a difference of $\approx 0.2$~dex seen in \bizarreone-\sunanalog\ pair is
not likely due to chemical inhomogeneity in the birth cloud.


%\begin{deluxetable*}{lccc}
  \tablecaption{Astrometric and spectroscopic measurements of the pair
  \label{tab:kk}}
\tablehead{
  \colhead{Name} & \colhead{HD 240429} & \colhead{HD 240430} & \colhead{Uncertainties}
}
\startdata
Sp Type                             & G2                & G0                &       \\
$T_\mathrm{eff}$ [K]                & 5878              & 5803              & 25    \\
$\log{g}$                           & 4.43              & 4.33              & 0.028 \\
$v\sin{i}$                          & 1.1               & 2.5               &       \\
$[\elem{Fe}/\elem{H}]$              & 0.01              & 0.20              & 0.010 \\
$v_r$ [\kms]                        & $-21.2$           & $-21.2$           & 0.5   \\
$\varpi$ \tablenotemark{a}           & $9.35 \pm 0.24$   & $9.41 \pm 0.25$   &       \\
$\mu_\alpha^*$ \tablenotemark{a}     & $89.25 \pm 0.66$  & $89.41 \pm 0.69$  &       \\
$\mu_\delta$ \tablenotemark{a}       & $-29.68 \pm 0.54$ & $-30.12 \pm 0.52$ &       \\
\hline 
\multicolumn{4}{c}{$T_c < 1200$~K} \\
\hline 
$A(\elem{Li})$ \tablenotemark{b}     & $2.25$            & $2.75$            &       \\
$\elemH{C}$                         & $0.00$            & $0.09$            & 0.026 \\
$\elemH{N}$                         & $-0.06$           & $-0.01$           & 0.042 \\
$\elemH{O}$                         & $0.01$            & $0.09$            & 0.036 \\
$\elemH{Na}$                        & $-0.06$           & $-0.04$           & 0.014 \\
$\elemH{Mn}$                        & $-0.03$           & $0.00$            & 0.020 \\
\hline 
\multicolumn{4}{c}{$T_c > 1200$~K} \\
\hline 
$\elemH{Mg}$                        & $0.01$            & $0.19$            & 0.012 \\
$\elemH{Al}$                        & $0.01$            & $0.21$            & 0.028 \\
$\elemH{Si}$                        & $0.00$            & $0.16$            & 0.008 \\
$\elemH{Ca}$                        & $0.02$            & $0.23$            & 0.014 \\
$\elemH{Ti}$                        & $0.02$            & $0.20$            & 0.012 \\
$\elemH{V}$                         & $0.02$            & $0.20$            & 0.034 \\
$\elemH{Cr}$                        & $0.01$            & $0.17$            & 0.014 \\
$\elemH{Fe}$                        & $0.01$            & $0.20$            & 0.010 \\
$\elemH{Ni}$                        & $-0.01$           & $0.21$            & 0.014 \\
$\elemH{Y}$                         & $0.04$            & $0.26$            & 0.030
\enddata
\tablenotetext{a}{From \tgas.}
\tablenotetext{b}{Absolute abundances from \citealt{jmlithium}}
\tablecomments{
  All values are from \citealt{2016ApJS..225...32B} unless otherwise noted.
}
\end{deluxetable*}


\end{document}
