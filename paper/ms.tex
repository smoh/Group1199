\documentclass[modern, letterpaper]{aastex61}
% \documentclass[twocolumn]{aastex61}

% to-do list
% ----------
% - Add items here.

% style notes
% -----------
% - This file generates by Makefile; don't be typing ``pdflatex'' or some bullshit.
% - Line break between sentences to make the git diffs readable.
% - Use \, as a multiply operator.
% - Reserve () for function arguments; use [] or {} for outer shit.
% - Use \sectionname not Section, \figname not Figure, \documentname not Article or Paper or paper.
% - Use "comoving" instead of "co-moving".
% - Use "phase space" not "phase-space", "phase-space coordinates" not "phase space coordinates".
% - Write elements as \elem{X}.

% packages
\definecolor{cbblue}{HTML}{3182bd}
\usepackage{microtype}  % ALWAYS!
\usepackage{amsmath,amssymb,natbib}
\usepackage[flushleft]{threeparttable}
\hypersetup{backref,breaklinks,colorlinks,urlcolor=cbblue,linkcolor=cbblue,citecolor=gray}
\graphicspath{{figures/}}
\bibliographystyle{aasjournal}

% define macros for text
\newcommand{\project}[1]{\textsl{#1}}
\newcommand{\acronym}[1]{{\small{#1}}}
\newcommand{\gaia}{\project{Gaia}}
\newcommand{\rave}{\project{\acronym{RAVE}}}
\newcommand{\apogee}{\project{\acronym{APOGEE}}}
\newcommand{\tmass}{\project{\acronym{2MASS}}}
\newcommand{\documentname}{Article}
\newcommand{\sectionname}{Section}
\newcommand{\figname}{Figure}
\newcommand{\eqname}{Equation}
\newcommand{\dr}{\acronym{DR1}}
\newcommand{\tgas}{\acronym{TGAS}}
\newcommand{\etal}{\textit{et al}.}
\newcommand*\elem[1]{\ensuremath{\mathrm{#1}}}
\newcommand*\elemH[1]{\ensuremath{[\mathrm{#1}/\elem{H}]}}
\newcommand*\teff{\ensuremath{T_\mathrm{eff}}}
\newcommand*\logg{\ensuremath{\log{g}}}
\newcommand*{\feh}{\ensuremath{\elemH{Fe}}}
\newcommand{\sunanalog}{\text{Krios}}
\newcommand{\bizarreone}{\text{Kronos}}
\newcommand{\Tcondens}{\ensuremath{T_C}}
\newcommand{\mearth}{\ensuremath{M_\oplus}}
\newcommand{\mjupiter}{\ensuremath{M_\mathrm{Jup}}}

% define macros for math
\newcommand{\given}{\,|\,}
\newcommand{\normal}{{\mathcal{N}}}
\newcommand{\dd}{\mathrm{d}}
\newcommand{\transp}[1]{{#1}^{\!\mathsf{T}}}
\newcommand{\inv}[1]{{#1}^{-1}}
\newcommand{\bs}[1]{\boldsymbol{#1}}
\newcommand{\vperp}{\bs{v}^\perp}
\newcommand{\propm}{\bs{\mu}}
\newcommand{\mat}[1]{\mathbf{#1}}
\renewcommand{\vec}[1]{\bs{#1}}
\newcommand{\kms}{\ensuremath{\rm km~s^{-1}}}
\newcommand{\msun}{\ensuremath{{\mathrm M}_\odot}}
\newcommand{\pc}{{\rm pc}}
\newcommand{\data}{\mathrm{data}}
\newcommand{\snr}{[S/N]_\varpi}
\newcommand{\eye}{\mathbb{I}}
\newcommand{\absdvtan}{\ensuremath{|\Delta\vec v_\mathrm{t}|}}
\newcommand{\estimates}{\ensuremath{\{\hat{\varpi_i},\hat{\mu_{\alpha,i}},\hat{\mu_{\delta,i}},\hat{v_{r,i}}\}}}

\newcommand{\todo}[1]{{\color{blue}TODO: #1\\}}
\newcommand*{\askjmb}[1]{{\bf #1}}

\renewcommand\tablename{Table}

\begin{document}\sloppy\sloppypar\raggedbottom\frenchspacing % trust me

\title{
  Kronos \& Krios:
  Evidence for accretion of a massive, rocky planetary system
  in a comoving pair of solar-type stars
}

\author[0000-0001-7790-5308]{Semyeong Oh}
\affil{Department of Astrophysical Sciences, Princeton University, 4 Ivy Lane,
  Princeton, NJ 08544, USA}
\affil{To whom correspondence should be addressed: \texttt{semyeong@astro.princeton.edu}}

\author[0000-0003-0872-7098]{Adrian M. Price-Whelan}
\affil{Department of Astrophysical Sciences, Princeton University, 4 Ivy Lane,
  Princeton, NJ 08544, USA}

\author[0000-0002-9873-1471]{John M. Brewer}
\affil{Department of Astronomy, Yale University, 260 Whitney Ave,
  New Haven, CT 06511, USA}
\affil{Department of Astronomy, Columbia University, 550 West 120th Street,
  New York, NY 10027, USA}

\author[0000-0003-2866-9403]{David W. Hogg}
\affil{Center for Computational Astrophysics, Flatiron Institute, 162 Fifth Ave,
  New York, NY 10010, USA}
\affil{Center for Cosmology and Particle Physics,
  Department of Physics, New York University, 726 Broadway,
  New York, NY 10003, USA}
\affil{Center for Data Science, New York University, 60 Fifth Ave,
  New York, NY 10011, USA}
\affil{Max-Planck-Institut f\"ur Astronomie, K\"onigstuhl 17, D-69117 Heidelberg}

\author{David N. Spergel}
\affil{Department of Astrophysical Sciences, Princeton University, 4 Ivy Lane,
  Princeton, NJ 08544, USA}
\affil{Center for Computational Astrophysics, Flatiron Institute, 162 Fifth Ave,
  New York, NY 10010, USA}
\author{Justin Myles}
\affil{Department of Astronomy, Yale University, 260 Whitney Ave, New Haven, CT 06511, USA}

\begin{abstract}
  We report and discuss the discovery of a comoving pair of bright solar-type
  stars, HD~240430 and HD~240429, with a significant difference in their
  chemical abundances.
  The two stars have an estimated separation of $\approx 0.6$~pc at a distance of
  $r\approx 100~\pc$ with nearly identical three-dimensional velocities,
  %($|\Delta \bs{v}| < XX~\kms$ with 95\% confidence),
  as inferred from \gaia\ \tgas\ parallaxes and proper motions, and
  high-precision radial velocity measurements.
  Stellar parameters determined from high-resolution Keck HIRES spectra
  indicate that both stars are $\sim 4$~Gyr old.
  The more metal-rich of the two, HD~240430, shows an enhancement of refractory
  ($\Tcondens>1200$~K) elements by $\approx 0.2$~dex but not as large
  an enhancement of (moderately) volatile elements ($\Tcondens<1200$~K; \elem{C},
  \elem{N}, \elem{O}, \elem{Na}, and \elem{Mn}).
  This is the largest and the most significant chemical difference found in
  a wide binary pair yet.
  Additionally, it shows anomalously high surface lithium abundance
  ($A(\elem{Li}=2.75$), higher by $0.5$~dex than its companion.
  The proximity in phase-space and ages between the two stars suggests that
  they formed together with the same composition, at odds with the observed
  differences in metallicity and abundance patterns.
  We therefore suggest that the star HD~240430, ``Kronos'', recently accreted
  $\approx 20$~\mearth\ of rocky material after birth selectively enhancing the
  refractory elements in its surface and convective envelope.
\end{abstract}

\keywords{
  binaries: visual
  ---
  planet-star interactions
  ---
  stars: abundances
  ---
  stars: formation
  ---
  stars: individual (HD~240430, HD~240429)
  ---
  stars: solar-type
}

\section{Introduction} % (fold)
\label{sec:introduction}

Wide binary stars are valuable tools for studying star and planet formation as
well as Galactic dynamics and chemical evolution.
In the context of studying the evolution of the Milky Way, they are useful for
two main reasons.
First, because wide binaries are weakly bound systems that may be tidally
disrupted by, e.g., field stars, molecular clouds, or the Galactic tidal field,
their statistics can be informative of the Galactic mass distribution.
For example, the separation distribution of halo binaries has been used to
constrain the mass of massive compact halo objects
(\citealt{Yoo:2004aa,Quinn:2009,Allen:2014}).
They can also be used to test the ``chemical tagging'' hypothesis that stars
from the same birthplace may be traced back using detailed chemical abundance
patterns as birth tags (\citealt{2002ARA&A..40..487F}).
While any multiple-star system including massive open clusters can be used to
test the hypothesis, wide binaries have the advantage of being extremely
abundant, rendering their statistics a meaningful indication of whether the
hypothesis works.

Binary stars that form from the same birth cloud start with nearly identical
composition.
A differential analaysis of the chemical composition of binary stars can reveal
their history through the chemical signatures related to planet formation or
accretion regardless of the Galactic chemical evolution.
\askjmb{
For example, \citealt{Fischer:2005aa} showed that the higher metallicity in
stars hosting close-in giant planets compared to those without likely promotes
the formation of giant planets by increasing the availability of small particle
condensates that form planetesimals.
}
On the other hand, if host stars are polluted after their birth by rocky
planetary material with high refractory-to-volatile ratio, the convective
envelope of the stars may be enhanced in refractory elements (e.g., \elem{Fe})
compared to their initial state (e.g., \citealt{Pinsonneault:2001aa}).
Thus, differences in planet formation or accretion in two otherwise identical
stars may imprint differences in chemical abundances that depend on the
condensation temperature (\Tcondens).

High resolution spectroscopic studies of binary star systems hosting at least
one planet
(\citealt{Teske:2013aa,Mack:2014aa,Liu:2014aa,Teske:2015aa,Saffe:2015aa,
  Ramirez:2015aa,Biazzo:2015aa,Mack:2016aa,Teske:2016aa,Teske:2016ab})
have yielded varied results:
while some systems appear to have undetectable differences
(see also \citealt{Desidera:2004aa,Gratton:2001aa}), other
studies have reported a \Tcondens-dependent difference in abundance
with higher-\Tcondens\ elements showing larger differences.
A possible explanation for the increasing abundance difference with
condensation temperature is that more yet-undiscovered rocky planets formed
around the relatively refractory-poor star
(\citealt{Ramirez:2015aa,Biazzo:2015aa}).
Alternatively, late time accretion of refractory-rich planetary material to the
relatively refractory-rich star can also produce the trend.
The observed differences are $\lesssim 0.1$~dex even in the most dramatic case,
and often at $\approx 0.05$~dex level making them challenging to detect even
with a careful analysis of high-resolution, high signal-to-noise ratio spectra,
and differential analyses of two stars that are very similar in their
stellar parameters.

Spectral analysis of polluted white dwarfs (WDs) currently provides the
strongest evidence for accretion of planetary material by a host star
(\citealt{2003ApJ...596..477Z,2010ApJ...722..725Z,2014A&A...566A..34K};
see \citealt{2016NewAR..71....9F} for review).
Because the gravitational settling times of elements heavier than \elem{He} in
atmosphere is much shorter than the WD cooling time
(\citealt{1986ApJS...61..197P}), detection of metals likely indicates the
presence of a reservoir of dusty material around the WD.
Indeed, many of the polluted WDs host a dusty debris disk detected in the
infrared (\citealt{1987Natur.330..138Z,1990ApJ...357..216G,2005ApJ...635L.161R,
  2009ApJ...694..805F,2006ApJ...646..474K}).
Some of the most dramatically polluted WDs show
surface abundances closely matched by rocky planetary material
with, e.g., bulk Earth composition, strongly arguing
that the disk formed from tidally disrupted minor planets
(\citealt{Zuckerman:2007aa,Klein:2010aa}).
Recently, transit signals from small bodies orbiting around a polluted WD
have been detected by \project{Kepler} adding further support to the picture
(\citealt{2015Natur.526..546V}).

Here, we report and discuss the discovery of a comoving pair of G stars,
HD~240430 and HD~240429, with unusual chemical abundance differences that
strongly suggest accretion of rocky planetary material by one of the two stars,
HD~240430.
Throughout the \documentname, we nickname the two stars \bizarreone\
(HD~240430) and \sunanalog\ (HD~240429).
In Greek mythology, Kronos and Krios were sons of Uranos and Gaia.
Kronos notoriously devoured all of his children (except Zeus)
to prevent the prophecy that one day he will be overthrown by them.
We use the following convention for chemical abundances of stars: \elemH{X} is
the log ratio of number density of element \elem{X} to \elem{H} relative to the
solar value.
The absolute abundance of \elem{X} is $A(\elem{X}) = 12 + \log_{10}
(n_\elem{X}/n_\elem{H})$ where $n_\elem{X}$ is the number density of element
\elem{X}.
In \sectionname~\ref{sec:data} we present the astrometric and spectroscopic data
about the two stars relevant to the present discussion.
In \sectionname~\ref{sec:discussion} we discuss possible interpretations of
the abundance difference between the pair.
We summarize our discussions in \sectionname~\ref{sec:summary}.

\subsection{
  Review of Detailed Chemical Abundance Studies of Stars in Comoving Pairs}
\label{sub:review}

Anticipating the forthcoming discussion, we review and summarize a handful of
wide binary systems that have been studied in their detailed chemical
abundances so far with high-resolution spectroscopy.
These systems are HD~20782/HD~20781, HD~80606/HD~80607, XO-2N/XO-2S, HAT-P-1,
WASP-94A/WASP94-B, and HD~133131A/HD~133131B.
We focus on key characteristics of stars and planets, and interpretations of
any trend in $\Delta\elemH{X}$ with \Tcondens.

\todo{smoh: discussion of HD~1333131AB and 16 Cyg is missing, but will be added}

{\bf 16 Cygni A/B:}
The chemical composition of this well know pair of solar-type stars (G1.5/G3)
has been studied many times.
The hotter star 16 Cyg A has no detected planets, but a M dwarf companion $\sim
70$~AU away in projected separation which is probably physically associated
(\citealt{2002ApJ...581..654P}), and may have affected planet formation process
around the star (\citealt{1996ApJ...458..312J,2005MNRAS.363..641M}).
The other star, 16 Cyg B, hosts a giant planet on an eccentric orbit ($e=0.63$,
\citealt{1997ApJ...483..457C}).
While past measurements of metallicity and abundance difference between the two stars
claimed conflicting results (\citealt{2001ApJ...553..405L,2011ApJ...737L..32S}),
recent studies using high quality spectra consistently reported that
A is more metal rich than B by $\approx 0.04 \pm 0.005$~dex.
However, there is still a disagreement between studies on
whether abundance differences shows a correlation with $\Tcondens$ as well as
its interpretation.

{\bf HD~20782/HD~20781:}
Two common proper motion G dwarf stars (G2/G9.5) with a projected separation of
$\sim9000$~AU (corresponding to 4.2\arcmin\ sky separation) and solar metallicity
host close-in giant planets.
HD~20782 hosts a Jupiter-mass planet on a very eccentric ($e\approx 0.97$)
orbit with a pericenter distance of 1.4~AU while HD~20781 hosts two
Neptune-mass planets within 0.3~AU with moderately high eccentricity
($e\sim0.1-0.3$).\footnote{
  The two stars were monitored by \project{HARPS} campaign, and it has recently
  been reported by \citealt{2017arXiv170505153U} that HD~20781 hosts four
  planets between $M\sin(i)\approx 0.006-0.04$~\mjupiter\ with $e \le 0.11$
  within $\approx 0.35$~AU.}
The measured abundances of 15 elements between the two stars are consistent
with each other (\citealt{Mack:2014aa}).
However, \citealt{Mack:2014aa} argued that there is a moderately significant
($\sim 2\sigma$) positive slope of $\approx 10^{-5}$~dex\,K$^{-1}$ with
increasing \Tcondens\ for $\Tcondens>900$~K elements (namely, Na, Mn, Cr, Si,
Fe, Mg, Co, Ni, V, Ca, Ti, Al, Sc leaving out C and O of their measurements) in
the abundances of each star {\it individually}.
They suggest that this slope is evidence that the stars accreted
$10-20$~\mearth\ or \elem{H}-depleted rocky material during giant planet
migration.

{\bf HAT-P-1:}
This pair of G0 stars separated by 11\arcsec\ with $\feh\approx0.15$ has
different planetary systems:
the secondary star is known to host one transiting giant planet
while no planet has been discovered around the primary star.
The two stars are identical
in metallicities and abundances for 23 elements measured with
the mean error of $0.013$~dex (\citealt{Liu:2014aa}).
Thus, it seems that the presence of close-in giant planet does not necessarily
lead to atmospheric pollution of its host star.

{\bf HD~80606/HD~80607:}
Similar to HAT-P-1, no significant chemical difference is found between two
common proper motion G5 stars with super-solar metallicity ($\feh \approx
0.35$): HD~80606 which hosts a very eccentric ($e\approx0.94$) giant planet and
HD~80607 which has no detected planets (\citealt{Saffe:2015aa,Mack:2016aa}).

{\bf XO-2N/XO-2S:}
This pair of G9 stars with super-solar metallicity ($\feh \gtrsim 0.35$)
has received much attention due to possibly the most robust and significant
difference between the component stars.
XO-2N hosts a giant planet while XO-2S is known to host two giant planets with
masses $0.26 \mjupiter$ and $1.37 \mjupiter$ on moderately eccentric ($\approx
0.15$) orbits at $<0.5$~AU.
%TODO: check following with Biazzo+
All measured elements including \elem{Fe} are enhanced in XO-2N relative to
XO-2S, and the difference shows a significant correlation with \Tcondens\
(\citealt{Teske:2015aa,Ramirez:2015aa,Biazzo:2015aa} although see also
\citealt{Teske:2013aa}).
At low \Tcondens, volatile elements differ by $0.015$~dex while the range of
difference spans upto $0.1$~dex at $\Tcondens>1600$~K.

\citealt{Ramirez:2015aa} argued that the {\it depletion} of volatile elements
in XO-2S relative to XO-2N is plausibly due to the presence of {\it more} gas
giant planets around XO-2S, following a similar interpretation of
\citealt{Melendez:2009aa} of the trend between solar twins and the Sun.
In this scenario, forming planets in the protoplanetary disk are supposed to
``lock'' heavier elements to the core of gas giant planets as well as some
volatile elements in their envelopes. Then, the host star will accrete {\it
  gas} depleted in metals compared to their protostellar condition.
By tuning the metal content ($Z/X$) of gas giant planets and when the accretion
of gas occurs in terms of how much mass is in the convective envelope of the
host star, they can match the mass difference of gas giant planets around each
star, which is at least $1~\mjupiter$.
On the positive correlation of $\Delta\elemH{X}$ with \Tcondens,
\citealt{Ramirez:2015aa} estimated that $20~\mearth$ of equal-part mixture of
Earth and CM chondrite-like material can explain the trend either as refractory
{\it depletion} in XO-2S due to {\it more} rocky planets around XO-2S, or
refractory enhancement in XO-2N by late time accretion in which case the
opposite would be true.

{\bf WASP-94A/B:}
Each star in a pair of F8 and F9 stars with super-solar metallicity
($\feh\approx 0.3$) hosts a hot Jupiter.
The planet around WASP-94A is transiting with a misaligned, probably retrograde
circular ($e<0.13$) orbit, while that hosted by WASP-94B is a little more
massive by $\sim 0.15$~\mjupiter\ and closer in, aligned with the host star.
WASP-94A shows a depletion of $0.02$~dex in volatile and moderately volatile
elements ($\Tcondens < 1200$~K) and an enhancement of $0.01$~dex in refractory
elements ($\Tcondens>1200$~K) relative to WASP-94B, with a claimed median
uncertainty of $0.006$~dex among all elements
resulting in a statistically significant non-zero slope between
$\Delta\elemH{X}$ and $\Tcondens$ (\citealt{Teske:2016aa}).\footnote{
  %TODO: double check the following statement
  Note that the condensation
  temperature $\Tcondens$ used is for solar system composition
  gas, which can differ from that of higher metallicity gas.}
Multiple possibilities related to the formation and evolution
of planetary systems around each star as well as causes unrelated to planets
such as dust cleansing during the fully convective phase or different rotation
and granulation between the stars were considered, but none was favored.

{\bf $\zeta^1/\zeta^2$ Reticuli (HD 20807/HD 20766):}
With a projected separation of $\approx 3700$~AU, both solar-type stars in this pair have
no detected planets.
However, $\zeta^2$ hosts a debris disk detected via infrared excess
(\citealt{2008ApJ...674.1086T}) as well as direct imaging
(\citealt{2010A&A...518L.131E}).
Both stars have super-solar metallicity of $\approx 0.2$~dex.
A differential abundance analysis using high-resolution spectra
shows that $\zeta^1$ is more metal rich than $\zeta^2$ by $\sim 0.02 \pm 0.003$~dex,
and that there is a positive slope between the abundance differences of 24 species
and $\Tcondens$.
A possible explanation proposed is that the relative lack of refractory elements
in $\zeta^2$ is because they are locked up in rocky bodies
that make up its debris disk (\citealt{2016A&A...588A..81S}).

{\bf HD~133131A/B:}

\section{Data}
\label{sec:data}

\begin{deluxetable*}{lccc}
  \tablecaption{Astrometric and spectroscopic measurements of the pair
  \label{tab:kk}}
\tablehead{
  \colhead{Name} & \colhead{HD 240429} & \colhead{HD 240430} & \colhead{Uncertainties}
}
\startdata
Sp Type                             & G2                & G0                &       \\
$T_\mathrm{eff}$ [K]                & 5878              & 5803              & 25    \\
$\log{g}$                           & 4.43              & 4.33              & 0.028 \\
$v\sin{i}$                          & 1.1               & 2.5               &       \\
$[\elem{Fe}/\elem{H}]$              & 0.01              & 0.20              & 0.010 \\
$v_r$ [\kms]                        & $-21.2$           & $-21.2$           & 0.5   \\
$\varpi$ \tablenotemark{a}           & $9.35 \pm 0.24$   & $9.41 \pm 0.25$   &       \\
$\mu_\alpha^*$ \tablenotemark{a}     & $89.25 \pm 0.66$  & $89.41 \pm 0.69$  &       \\
$\mu_\delta$ \tablenotemark{a}       & $-29.68 \pm 0.54$ & $-30.12 \pm 0.52$ &       \\
\hline 
\multicolumn{4}{c}{$T_c < 1200$~K} \\
\hline 
$A(\elem{Li})$ \tablenotemark{b}     & $2.25$            & $2.75$            &       \\
$\elemH{C}$                         & $0.00$            & $0.09$            & 0.026 \\
$\elemH{N}$                         & $-0.06$           & $-0.01$           & 0.042 \\
$\elemH{O}$                         & $0.01$            & $0.09$            & 0.036 \\
$\elemH{Na}$                        & $-0.06$           & $-0.04$           & 0.014 \\
$\elemH{Mn}$                        & $-0.03$           & $0.00$            & 0.020 \\
\hline 
\multicolumn{4}{c}{$T_c > 1200$~K} \\
\hline 
$\elemH{Mg}$                        & $0.01$            & $0.19$            & 0.012 \\
$\elemH{Al}$                        & $0.01$            & $0.21$            & 0.028 \\
$\elemH{Si}$                        & $0.00$            & $0.16$            & 0.008 \\
$\elemH{Ca}$                        & $0.02$            & $0.23$            & 0.014 \\
$\elemH{Ti}$                        & $0.02$            & $0.20$            & 0.012 \\
$\elemH{V}$                         & $0.02$            & $0.20$            & 0.034 \\
$\elemH{Cr}$                        & $0.01$            & $0.17$            & 0.014 \\
$\elemH{Fe}$                        & $0.01$            & $0.20$            & 0.010 \\
$\elemH{Ni}$                        & $-0.01$           & $0.21$            & 0.014 \\
$\elemH{Y}$                         & $0.04$            & $0.26$            & 0.030
\enddata
\tablenotetext{a}{From \tgas.}
\tablenotetext{b}{Absolute abundances from \citealt{jmlithium}}
\tablecomments{
  All values are from \citealt{2016ApJS..225...32B} unless otherwise noted.
}
\end{deluxetable*}


\begin{figure}[htbp]
  \begin{center}
    \includegraphics[width=\linewidth]{dx_dv_posterior.pdf}
  \end{center}
  \caption{%
    Differences in posterior samples over Galactocentric phase-space coordinates
    for the two stars \sunanalog\ and \bizarreone.
    \label{fig:dxdv}}
\end{figure}

\begin{figure}[htpb]
  \centering
  \includegraphics[width=0.9\linewidth]{abundances.pdf}
  \caption{Abundances of the comoving pair, \sunanalog\ (blue) and \bizarreone\
    (red). Lines are drawn for each star only to guide the eye. \bizarreone\ is
    enhanced in \elem{Fe} by $\approx 0.2$~dex relative to \sunanalog\ along
    with \elem{Mg}, \elem{Al}, \elem{Si}, \elem{Ca}, \elem{Ti}, \elem{V},
    \elem{Cr}, \elem{Ni}, \elem{Y} yet not in \elem{C}, \elem{N}, \elem{O},
    \elem{Na}, and \elem{Mn}.
  }
  \label{fig:abundances}
\end{figure}

\begin{figure}[htpb]
  \centering
  \includegraphics[width=0.95\linewidth]{spec1.pdf}
  \caption{Selective segments of the spectra of \sunanalog\ and \bizarreone.
    Alternating sets of two rows show
    the continuum-normalized data and model in the upper panel,
    and the ratio (\bizarreone/\sunanalog) of data (gray) and model (black)
    in the lower panel.
  }
  \label{fig:spec1}
\end{figure}

\begin{figure}[htpb]
  \centering
  \includegraphics[width=0.95\linewidth]{spec2.pdf}
  \caption{Same as \figname~\ref{fig:spec1}
    but for smaller portions of spectra at longer wavelengths that are
    not dominated by \elem{Fe}.
    We mark elements that give rise to strong absorption lines.
    Note that the lines of \elem{Na} and \elem{O}, which are under-enhanced
    in \bizarreone\ relative to \elem{Fe} or other refractory elements,
    show weaker residuals.
  }
  \label{fig:spec2}
\end{figure}

\begin{figure}[htpb]
  \centering
  \includegraphics[width=0.6\linewidth]{spec_lithium.pdf}
  \caption{Lithium lines in the spectra of \bizarreone\ and \sunanalog.
    This line is studied in Myles \etal\ (in prep.).
    Line legends are the same as in \figname~\ref{fig:spec1}.
  }
  \label{fig:spec_lithium}
\end{figure}

\begin{figure}[htpb]
  \centering
  \includegraphics[width=0.95\linewidth]{deltaXH_elem_violins.pdf}
  \caption{Abundance difference in this pair and other twin-like
    ($\Delta T_\mathrm{eff}\lesssim 100$~K) wide binaries in
    \citealt{2016ApJS..225...32B}.
    The differences in other pairs are small ($<0.05$~dex)
    for all elements except \elem{N} and \elem{O} which are the most
    uncertain, making the difference of $\approx 0.2$~dex seen in
    \bizarreone-\sunanalog\ rare.
    Additionally, we show the distribution of abundance differences
    between field stars with similar metallicity difference
    ($\Delta[\elem{Fe}/\elem{H}] \approx 0.2$)
    as violins with medians indicated by black line segments.
    These are random pairings of single stars in
    in \citealt{2016ApJS..225...32B} at two metallicity bins,
    $-0.025 < \feh < 0.025$ (160 stars) and $0.175 > \feh > 0.225$ (137 stars),
    similar to \bizarreone\ and \sunanalog.
    The difference is always taken to be
    $\mathrm{higher}\,\feh - \mathrm{lower}\,\feh$.
    Thus, the narrower range of in $\Delta\feh$ is by construction.
    Random pairings of disk stars with similar $\Delta\feh$ usually show
    similar enhancement in all other elements
    unlike the pattern seen in \bizarreone-\sunanalog\ pair.
  }
  \label{fig:deltaXH}
\end{figure}

\sunanalog\ and \bizarreone\ were identified as a
candidate comoving star pair in our recent search for comoving stars using the
proper motions and parallaxes from the \tgas\ catalog, a component of \gaia\ \dr.
We refer the readers to this previous work (\citealt{2017AJ....153..257O}) for a
full explanation of the methodology behind this search and only include
a brief description here.
For a given pair, we compute a marginalized likelihood ratio between the
hypotheses (1) that a given pair of stars share the same 3D velocity vector,
and (2) that they have independent 3D velocity vectors, using only the
astrometric measurements from \tgas\ (parallaxes and proper motions).
We then select a sample of high-confidence comoving pairs by making a
conservative cut on this likelihood ratio.
In the resulting catalog of comoving pairs (\citealt{2017AJ....153..257O}),
the pair presented in this paper was assigned a group id of 1199,
and the marginalized likelihood ratio (Bayes factor)
between the two hypotheses is $\ln{\mathcal{L}_1/\mathcal{L}_2} = 8.52$,
well above the adopted cut value of 6.
The pair has also been previously recognized as a visual double star system
in Washington Double Star catalog (\citealt{2001AJ....122.3466M}).
We have checked that we do not find any possible additional comoving companions
by lowering the likelihood ratio cut for the stars around this pair.

In a separate effort to study detailed chemical abundances of potential
planet-hosting stars, high-resolution spectra of both stars were obtained using
the HIRES spectrograph on the Keck~I telescope, and analyzed
\citep{2016ApJS..225...32B}.
The spectral resolution is $R\approx 70000$ and the wavelength coverage is
$5164$--$7799$~\AA.
A typical signal-to-noise ratio in the spectral continuum is $>200$~per pixel.
The resulting measurements include elemental abundances for 15 chemical species
(C, N, O, Na, Mg, Al, Si, Ca, Ti, V, Cr, Mn, Fe, Ni, Y) as well as stellar parameters
and high precision radial velocities.
For the details of the spectral analysis, we refer the readers to
\citealt{2016ApJS..225...32B}.
Additionally, the \elem{Li} doublet at $6707.6$~\AA\
%--- clearly visible in both spectra (see \figname~\ref{fig:spec2}) ---
for this sample was investigated in a separate work (\citealt{jmlithium}).
We list all relevant astrometric and spectroscopic measurements
including the absolute abundance\footnote{
  The absolute abundance of element \elem{X} is
  $A(\elem{X}) = \log_{10} (n_\elem{X}/n_\elem{H})$
  where $n_\elem{X}$ is the number density of element \elem{X}.
}
of \elem{Li} for the two stars in Table~\ref{tab:kk}.

The projected separation between the pair is 1.9\arcmin\ ($\approx 0.01$~pc),
and the 3D separation is $\approx 0.6$~pc.
Although selected based only on their astrometry, the two stars
have identical radial velocities within uncertainties (Table~\ref{tab:kk}),
confirming that they are truly comoving.
Combining these precise radial velocities with the \gaia\ \tgas\ astrometry, we
can compare differences between the inferred 6D phase-space coordinates of the
two stars.
We start by generating posterior samples over the Heliocentric distance, $r$,
tangential velocities, $(v_\alpha, v_\delta)$, and radial velocity, $v_r$,
given the observed parallax, $\hat\pi$, and proper motions,
$(\hat\mu_{\alpha^*}, \hat\mu_\delta)$, and radial velocity, $\hat v_r$.
We assume the noise is Gaussian, and the radial velocity measurement is
uncorrelated with the astrometric measurements.
If we define
\begin{eqnarray}
  \vec{\hat y} &=&
      \transp{\left(
        \begin{array}{c@{\hspace{1em}} c@{\hspace{1em}} c@{\hspace{1em}} c}
          \hat\pi &
          \hat\mu_{\alpha^*} &
          \hat\mu_\delta &
          \hat v_r
        \end{array}
      \right)}\\
  \vec{y} &=&
      \transp{\left(
        \begin{array}{c@{\hspace{1em}} c@{\hspace{1em}} c@{\hspace{1em}} c}
          r^{-1} &
          r^{-1}\,v_\alpha &
          r^{-1}\,v_\delta &
          v_r
        \end{array}
      \right)}
\end{eqnarray}
then the likelihood is
\begin{equation}
  \vec{\hat y} \sim \mathcal{N}(\vec{y}, \mat{C})
\end{equation} where $\mat{C}$ is the covariance matrix.
We adopt a uniform space density prior for the distance and an isotropic
Gaussian for any velocity component, $v$, with a dispersion $\sigma_v=25~\kms$
\begin{eqnarray}
p(r) &=&
  \begin{cases}
    \frac{3}{r_{\rm lim}^3} \, r^2 & \text{if } 0 < r < r_{\rm lim}\\
    0              & \text{otherwise}
  \end{cases}\\
p(v) &=& \frac{1}{\sqrt{2\pi}\,\sigma_v} \,
  \exp\left[-\frac{1}{2} \, \frac{v^2}{\sigma_v^2} \right] \quad .
\end{eqnarray}
%
For each of the two stars, we use \project{emcee}
(\citealt{2013PASP..125..306F}) to generate posterior samples in $(r, v_\alpha,
v_\delta, v_r)$ by running 64 walkers for 4608 steps and discarding the first
512 steps as the burn-in period.
For each sample, we convert the heliocentric phase-space coordinates into
Galactocentric coordinates assuming that the Sun's position and velocity are
$\vec x_\odot = (-8.3,\,0,\,0)~{\rm kpc}$ and $\vec v_\odot =
(-11.1,\,244,\,7.25)~\kms$ \citep[e.g.,][]{Schonrich:2010, Schonrich:2012}.

\figurename~\ref{fig:dxdv} shows differences in posterior samples converted to
Galactocentric phase-space coordinates for the two stars.
The differences in velocities are consistent with zero.
For a 2~\msun\ binary system, the Jacobi radius in the Galactic neighborhood is
1.2~pc (\citealt{Jiang:2010aa}).
Thus, \bizarreone\ and \sunanalog\ are likely a bound system that formed coevally,
and we expect the two stars to have identical metallicities and abundance patterns.
However, one of the stars, \bizarreone\ is significantly more metal
rich than the other (by 0.2~dex $\approx 60\%$; \figname~\ref{fig:abundances}).
Moreover, not all elements are equally enhanced:
the abundances of \bizarreone\ show selective depletion in
\elem{C}, \elem{N}, \elem{O}, \elem{Na}, and \elem{Mn}
relative to \elem{Fe}.
\bizarreone\ also has a high surface \elem{Li} abundances, and the difference in
\elem{Li} abundance ($\approx 0.5$~dex) is the largest among all elements
measured.

The validity of the measured abundance differences is further demonstrated in
\figname~\ref{fig:spec1}, \ref{fig:spec2}, and \ref{fig:spec_lithium}, where we
show segments of the spectra and models of the two stars used to measure their
abundances (\citealt{2016ApJS..225...32B}).
As expected from their reported metallicity difference ($\Delta\feh \approx 0.2$),
the ratio of data and model between the two stars show significant
residuals of almost all metal line features, largely dominated by \elem{Fe}.
However, for lines of elements that are not as enhanced in \bizarreone\,
the residuals are much smaller in amplitude (\figname~\ref{fig:spec2}).
The \elem{Li} doublet, analyzed in a separate work (Myles \etal\ in prep.),
is clearly visible in the spectra of both stars, and is stronger in \bizarreone\
(\figname~\ref{fig:spec_lithium}).

We stress that none of the other four twin-like ($\Delta T_\mathrm{eff}
\lesssim 100$~K) wide binary pairs examined by \citealt{2016ApJS..225...32B}
show discrepancies in abundances between the stars at this level.
As shown in \figname~\ref{fig:deltaXH},
the differences in other pairs for all elements except \elem{N} and \elem{O},
which are also the most uncertain (\tablename~\ref{tab:kk}),
are less than $0.05$~dex, making \bizarreone-\sunanalog\ pair a significant outlier.
The statistical uncertainties for each parameter
presented in \tablename~\ref{tab:kk} from \citealt{2016ApJS..225...32B}
are estimated from repeated measurements of multiple spectra of the same stars.
We note that while there may be systematic uncertainties (bias) in the elemental
abundances of these two stars unconstrained by this procedure,
the systematic uncertainties for these two solar-type ``twin-like'' stars
with small differences in $T_\mathrm{eff}$ and $\log{g}$, if any, are unlikely to wash out
the observed abundance differences of $\approx 0.2$~dex.

\begin{figure}[htbp]
  \begin{center}
    \includegraphics[width=\linewidth]{orbits.pdf}
  \end{center}
  \caption{Left panel: Galactic orbits computed for \sunanalog\ (black) and the
    Sun (grey).
    For \sunanalog, the initial conditions are set to the median of the
    posterior samples over the phase-space coordinates.
    The orbits are computed by integrating backwards from the present-day
    positions for $2.5$~Gyr with a time step of $0.5$~Myr using the Leapfrog
    integration scheme implemented in \project{Gala} (\citealt{gala}).
    Right panel: distribution of maximum $z$-heights for orbits computed from
    all posterior samples.
  }
  \label{fig:orbit}
\end{figure}

\begin{figure}[htpb]
  \centering
  \includegraphics[width=0.95\linewidth]{TcRank_deltaXH_concise.pdf}
  \caption{Abundance differences of the \bizarreone-\sunanalog\ pair
    ranked by the condensation temperature of elements for solar composition gas
    from \citealt{2003ApJ...591.1220L}.
    The condensation temperature may be read from the gray line and right y-axis.
    We show three wide binary systems selected from the literature:
    HD~20782/1 (\citealt{Mack:2014aa}, $\feh\approx0$),
    XO-2N/S (\citealt{Biazzo:2015aa}, $\feh\approx0.35$),
    and WASP-94AB (\citealt{Teske:2016aa}, $\feh\approx0.3$).
    Locations of elements with at least one measurement from any study
    are indicated by a vertical line and its symbol.
    Note that often multiple values are reported for one element corresponding
    to different ionization states in equivalent width analyses.
    No other pair studied so far were shown to have such large difference
    in metallicity or sharp contrast between (moderately) volatile and
    refractory elements as \bizarreone-\sunanalog.
  }
  \label{fig:relabun_tcrank}
\end{figure}


\section{Discussion}
\label{sec:discussion}

We discuss the possible origins of the peculiar abundance differences of
\bizarreone-\sunanalog.
We first discuss the ages and coevality of the stars in this pair, and consider
both possibilities in which the two stars are or are not coeval.
Our favored scenario is discussed in the last subsection,
\sectionname~\ref{sub:accretion}.

% chance pair
%Given that their metallicities and abundance patterns are significantly
%different, one may simply conclude that the two stars are not related (coeval)
%but they merely happen to be comoving at such small separation ($\approx 0.6$~pc)
%by chance.
%The two stars are in the Galactic disk, and assuming certain velocity ellipsoid
%at the stars' location, one may compute the probability that a star moving at
%the mean velocity of the two stars would have a companion within $\Delta v = XX$~km/s.
%...

\subsection{Stellar Ages \& Coevality}
\label{sub:ages}

The fact that the two stars share nearly identical space motion
at a separation of $< 1$~pc strongly supports that the pair is coeval.
We therefore consider other age indicators apart from their phase-space
coordinates to assess the ages and coevality of the two stars.
First, given the precise measurements of $\log(g)$ and $T_\mathrm{eff}$,
we can constrain the ages of the two stars by
comparing these values to theoretical isochrones.
We perform isochrone fitting to these values using the Yale-Yonsei model
isochrones (\citealt{2013ApJ...776...87S}).
The input data are $\log(g)$, $T_\mathrm{eff}$, \feh,
parallaxes, $B$-band magnitudes, and their errors.
The best-fit isochrone ages of \sunanalog\ and \bizarreone\ are
$4.00_{-1.56}^{+1.51}$~Gyr and $4.28_{-1.03}^{+1.11}$~Gyr, respectively,
consistent with them being coeval.

The surface lithium abundance in a sun-like star decreases with its age due to
mixing induced by convection or rotation, which brings the lithium into the
interior ($T>2.5 \times 10^{6}$~K) where it will be destroyed by proton capture
burning.
Thus, surface lithium abundance is an indicator of stellar ages.
The absolute $\elem{Li}$ abundance of $2.25$~dex for \sunanalog\ implies an age
of $\lesssim 1$~Gyr according to the theoretical models tuned to explain the
solar lithium abundance and rotational profile (\citealt{2005Sci...309.2189C}).
The lithium abundance of \bizarreone\, $2.74$~dex, shows the largest difference
among all measured elements.
This translates to a $\sim 500$~Myr difference in age.
Given the overall higher metal abundances and the peculiar abundance patterns
in \bizarreone, it is unclear, however, whether this higher $\elem{Li}$
abundance means a younger age or something else.
For example, the presence of $\elem{Li}$-rich red giant stars has been
attributed to the engulfment of substellar companions such as gas giant planets
or brown dwarfs which may replenish $\elem{Li}$ (\citealt{Casey:2016aa}).

In fact, the surface lithium abundance is the only indication that points to
younger ages.
If the two stars are indeed only several hundred Myrs old at most,
they are expected to be part of a larger comoving group of stars.
However, as we mention above, there is no evidence in our search of comoving pairs
using \tgas\ that the two stars belong to a larger group of young stars.
Very young stars often show signs of activity such as
X-ray emission from magnetic activity, emission lines, or infrared excess due to
circumstellar disks (\todo{needs refernces}).
We have compiled GALEX, Tycho-2, 2MASS, and WISE photometry for these stars,
and found no evidence for indications of activity in their spectral energy
distributions.
The low $v\sin(i)$ values (\tablename~\ref{tab:kk}) also argue against very
young ages that would be inferred from the surface lithium abundances.
Finally, we computed the Galactic orbit of the pair using the median of the
posterior sample over the phase space coordinates of \sunanalog, in a Milky
Way-like gravitational potential (similar to \texttt{MWPotential2014} from
\citealt{Bovy:2015}) using \project{Gala} (\citealt{gala}).
The pair's fiducial orbit has a vertical action larger than the Sun, favoring
an older age (\todo{needs references for vertical action as age indicator}).
We therefore conclude that the two stars are most likely coeval, $\sim 4$~Gyr
old main sequence stars, and that their unusually high \elem{Li} abundance
requires an alternative explanation.


\subsection{Exchange Scattering}
\label{sub:exchange_scattering}

Although the data described above strongly suggests that the two stars are
coeval, we may still consider scenarios in which the two stars are not actually
born together.
Two stars unrelated at birth may end up in a binary system via a binary-single
scattering event that results in an exchange of binary members.
In order to estimate the rate at which any binary-single
event will produce a wide binary system such as \sunanalog\ and \bizarreone,
we may consider the rate at which this wide binary will scatter with a field star to
result in an exchange reaction.
The cross-section of exchange scattering for a binary with semi-major axis $a$ is
\begin{eqnarray}
  \sigma_\mathrm{ex} = \frac{640}{81} \pi a^{2} \left(\frac{v_i}{v_c}\right)^{-6}
  \label{eq:crosssection}
\end{eqnarray}
where $v_i$ is the incoming velocity, and $v_c$ is the critical velocity,
defined as
\begin{eqnarray}
  v_c^2 = G \frac{m_1 m_2 (m_1 + m_2 + m_3)}{m_3 (m_1 + m_2)} \frac{1}{a}\,\,.
\end{eqnarray}
\eqname~\ref{eq:crosssection} is appropriate when $v \gg 1$
(\citealt{Hut:1983aa,Hut:1983ab}), which is the case for wide binaries
scattering with field (disk) stars.
If we assume that field stars are made of solar mass stars with a constant
number density $n=1$~pc$^{-3}$, and the incoming velocity of field stars is
$10$~km\,s$^{-1}$, a lower limit to the velocity dispersions of disk stars in
any direction, the rate of exchange scattering is
\begin{eqnarray}
  n \sigma_\mathrm{ex} v_i = 6.82\times 10^{-8}\,\mathrm{Gyr}^{-1}
  \frac{n}{\mathrm{pc}^{-3}} \frac{\mathrm{pc}}{a}
  \left(\frac{10~\mathrm{km}\,\mathrm{s}^{-1}}{v_i}\right)^5\,,
\end{eqnarray}
which is low enough to be negligible.

In fact, any exchange scattering scenario is unsatisfactory as typical
field stars with similar metallicities as \bizarreone\ and \sunanalog\ is very
unlikely to show the abundance difference pattern observed.
In \figname~\ref{fig:deltaXH}, we compare the abundance difference of
\bizarreone-\sunanalog\ with the distribution of abundance differences,
$\Delta[\elem{X}/\elem{H}]$, between random pairings of two stars with similar
\feh\ as \bizarreone\ and \sunanalog.
We see that when a star is enhanced in $\elem{Fe}$ by $0.2$~dex,
all other elements are typically enhanced at a similar level, with some variations.
Specifically, for a typical star with $\feh \approx 0.2$~dex, we generally
expect $[\elem{Na}/\elem{Fe}] > 0$ and $[\elem{Mn}/\elem{Fe}] > -0.1$
(\citealt{Battistini:2015aa,Bensby:2003aa}) making the low
[\elem{Na}/\elem{Fe}] and [\elem{Mn}/\elem{Fe}] seen in \bizarreone\ very
unlikely to arise from variations in Galactic chemical evolution.

\todo{HOGG: to provide some theoretical account adding to the above empirical evidence.}

\subsection{Chemical Inhomogeneity in Star Formation}
\label{sub:chemical_inhomogeneity_in_star_formation}

Assuming that the two stars were born together, one possibility to
explain the chemical difference is that there was chemical inhomogeneity within
the birth cloud.
However, there is already ample evidence against such a scenario.
First, none of the other seven similar wide binaries examined in
\citealt{2016ApJS..225...32B} show the same level of differences in abundances.
Though there is generally a larger spread in $\elem{C}$, $\elem{N}$ and
$\elem{O}$, and some pairs show a difference in particular elements as large as
$\approx 0.15$~dex.
The median and maximum \feh\ difference between component stars in the other
seven pairs is $0.02$~dex and $0.09$~dex, respectively.
The differences are even smaller (maximum $\Delta\feh = 0.03$~dex) if we
compare only twin-like ($\Delta T_\mathrm{eff} \lesssim 100$~K) pairs
(\figname~\ref{fig:deltaXH}).
This is consistent with the findings of \citealt{Desidera:2004aa}, who examined
23 wide binaries of late F to K dwarfs, and found most pairs show a difference
in \feh\ less than $0.02$~dex and none larger than $0.07$~dex.
Similarly, \citealt{Gratton:2001aa} found that four out of six equal mass
binaries have the same chemical composition with $\gtrsim 0.01$~dex
uncertainties.
Thus, a difference of $\approx 0.2$~dex seen in \bizarreone-\sunanalog\ pair is
not likely due to chemical inhomogeneity in the birth cloud.

\begin{figure}[htpb]
  \centering
  \includegraphics[width=0.95\linewidth]{toycalc.png}
  \caption{
    Comparing the observed abundance difference vs. \Tcondens\ rank
    to the expected change in solar surface abundance after adding $20$~\mearth\ of
    material with bulk Earth composition (\citealt{mcdonough2001composition}).
    The assumed mass fraction in the convective zone is $0.03$.
    All metals are ranked by their \Tcondens\ for solar composition gas,
    and the condensation temperature may be read from the gray line and right y-axis,
    same as in \figname~\ref{fig:relabun_tcrank}.
    Locations of the elements measured for \bizarreone-\sunanalog\ pair are
    indicated by a vertical line and its symbol.
    The close match with the observed abundance difference in \bizarreone-\sunanalog\ pair
    suggests that the abundance difference may be due to accretion of
    $20$~\mearth\ of rocky planetary material.
    The element \elem{Li} is off the plot and indicated with a red arrow
    (see text for details).
  }
  \label{fig:toycalc}
\end{figure}

\subsection{Accretion of rocky planetary material}
\label{sub:accretion}

Another possibility that two coeval stars may end up with different surface
abundances is accretion of planetary material after birth.
Instabilities may develop in a multi-planet system due to its chaotic nature
which may lead to planet engulfment or ejection by the host star
(\todo{needs references}).
%TODO: kozai?
Indeed, it is an important goal of many exoplanet studies
to detect chemical signatures of planet formation or accretion,
distinguish them from Galactic chemical evolution, and
connect them to theories of evolution of planetary systems.
One approach that is free from confusion with Galactic chemical evolution
is to compare two almost identical stars in a wide binary system.
Assuming that the component stars were born together with identical
initial composition, we may see a difference in their surface abundances
if the two stars then accreted different amounts of planetary material.
The resulting abundance difference may depend on the condensation
temperatures of elements in the protoplanetary disks from which the accreted
planets formed, as their compositions depend on the radial temperature gradient
in the disk.
%TODO: cautions on condensation temperature
% 1. it is just a proxy
% 2. it is **equlibrium** condensation temperature

In Figure~\ref{fig:relabun_tcrank}, we show the abundance difference
between \bizarreone\ and \sunanalog\ ordered by the rank of \Tcondens\
of each element.
The equilibrium condensation temperatures for the composition of solar system
are taken from \citealt{2003ApJ...591.1220L} (Table~8).
The difference seen in \bizarreone-\sunanalog\ is
compared to HD~20781/2, XO-2N/S, WASP-94A/B in \figname~\ref{fig:relabun_tcrank}.
The metallicity difference of $\approx 0.2$~dex observed in this pair
is larger than the differences seen in any other pairs studied so far.
Strikingly, the five under-enhanced elements in \bizarreone\
relative to \sunanalog\ are the five most volatile in all elements measured.
The difference in \elem{Mn} ($\Tcondens = 1158$~K) and
\elem{Cr} ($\Tcondens = 1296$~K) suggests a break in $\Tcondens \approx 1200$~K.
This $\Tcondens$-dependent trend of $\Delta\elemH{X}$,
combined with the enhanced $\elem{Li}$ abundance ($A(\elem{Li}) = 2.75$),
strongly suggests that accretion of rocky material has occured in \bizarreone.

How much mass of rocky material is needed to explain an increment of
$\approx 0.2$~dex?
We carry out simple toy calculations of expected $\Delta\elemH{X}$
in a Sun-like star's atmosphere by adding certain mass of bulk Earth composition
under these simplifying assumptions:
\begin{itemize}
  \item The material added is instantly and completely mixed.
  \item The atmospheric composition that we measure is identical throughout
    the star's radiative and convective zone.
  \item The surface abundance of the star has been altered only by the
    accretion event(s).
\end{itemize}

We take the solar abundances, $\elemH{X}$, of element \elem{X}
(\citealt{Asplund:2009aa}) which can be converted to mass fraction as
\begin{equation}
  f_{X,\mathrm{photo}} = \frac{10^{\elemH{X}} m_X}{\Sigma_X 10^{\elemH{X}} m_X}
\end{equation}
where $m_X$ is the mass of each element in e.g., atomic mass unit.
Assuming that the added material has a total mass $M_\mathrm{add}$, and the
mass fraction in each element $f_{X,\mathrm{add}}$,
the abundance difference is
\begin{equation}
  \Delta\elemH{X} = \log_{10} \frac{f_{X,\mathrm{photo}}\,f_\mathrm{CZ}\,M_\mathrm{star}
    + f_{X,\mathrm{add}}\,M_\mathrm{add}}
    {f_{X,\mathrm{photo}}\,f_\mathrm{CZ}\,M_\mathrm{star}}
\end{equation}
where $f_\mathrm{CZ}$ is the fraction of the star's mass in the convective envelope.
We take the composition of bulk Earth from a chondritic model of the Earth
(\citealt{mcdonough2001composition}).
Similar calculations have been performed by \citet{Chambers:2010aa} and
\citet{Mack:2014aa,Mack:2016aa}.

Figure~\ref{fig:toycalc} shows the expected change of surface abundances of
metals in a Sun-like star after $20~\mearth$ of material with composition of
bulk Earth is added.
A volatility trend that more volatile (low \Tcondens) elements are more
depleted in the Earth relative to CI or other carbonaceous chondrites
has long been known (\citealt{mcdonough2001composition}).
This trend is presumed to be closely related to the formation of terrestrial
planets and, in particular to the radial temperature gradient in a
protoplanetary disk.
The trend resulting from adding $20~\mearth$ of bulk Earth
provides an overall good match to the observed $\Delta\elemH{X}$,
suggesting that the refractory-enhanced star, \bizarreone\,
accreted $20~\mearth$ more of rocky planetary material than \sunanalog.

What about \elem{Li}?
The element \elem{Li} is worth special attention in the context of the
accretion scenario.
Because Li is present in either carbonaceous chondrites or bulk Earth with
a concentration of $1-1.5$~ppm in mass (\citealt{mcdonough2001composition}),
but is depleted quickly within the first couple of Gyrs on the surface of a
Sun-like star, accretion of either material at later times will significantly
replenish the lithium on the star's surface.
For the present-day Sun, the accretion of $20~\mearth$ of bulk Earth-like
material would result in $\Delta\elemH{Li} \approx 1.6$~dex (which
is indicated as an upward arrow in \figname~\ref{fig:toycalc}).
Incidentally, the star under examination, \bizarreone, has an age (informed by
stellar parameters) very close to the Sun ($4.28_{-1.03}^{+1.11}$~Gyr).
Thus, as long as we believe the Sun's present-day surface \elem{Li} abundance
to be ordinary, the accretion of $20$~\mearth\ by \bizarreone\ is consistent with $1.6$~dex
enhancement in its \elem{Li}.
This is exactly what we find: the \elem{Li} abundance of \bizarreone\ is
$A(\elem{Li}) = 2.75$ (Table~\ref{tab:kk}, \citealt{jmlithium})
approximately $1.65$~dex higher than the solar value of $1.1$~dex
(\todo{needs reference}).
Furthermore, this implies that the accretion event should have happened very
recently, or at least within the \elem{Li} depletion time ($\lesssim 1$~Gyr).

What about \sunanalog?
Considering its age of $4.00_{-1.56}^{+1.51}$~Gyr, \sunanalog\ is also enhanced
in \elem{Li} compared to other stars of similar ages by $\approx 1$~dex.
This enhancement puts an upper limit on the accreted mass to be
$\approx 4$~\mearth\ assuming \elem{Li} concentration of $1.1$~ppm
(\citealt{mcdonough2001composition}).
Unlike \bizarreone, we cannot know the pre-accretion abundances of
\sunanalog.\footnote{
  Strictly speaking, we can never know the pre-accretion abundances
  for \bizarreone\ either.
  We only have an {\it approximate} idea that this is not far from a solar-twin
  star \sunanalog, and because the deviation from this ``anchor'' star is
  large, it is reasonable to consider accretion by the Sun.
  In reality, \sunanalog\ may as well have had its own accretion history,
  which indeed seems to be the case according to its \elem{Li} abundance.
}
However, we note that the span of abundance difference between highly volatile
elements (\elem{C}, \elem{N}, \elem{O}, \elem{Na}) and refractory elements such
as \elem{Fe} expected from accreting $4$~\mearth\ to the Sun is $\approx
0.05$~dex.
This is comparable to the abundance difference between N, Na and Fe in
\sunanalog.
It is also interesting that \sunanalog\ shows a deficit in the same volatile
and moderately volatile elements (\elem{C}, \elem{N}, \elem{O}, \elem{Na}, and
\elem{Mn}) relative to Fe as in \bizarreone.
Thus, we conclude that \sunanalog\ is also likely to have had a similar
accretion event, but the amount of accretion was at least 5 time
smaller than \bizarreone.

%TODO: augment this paragraph
% - more geochemical study references
We stress that while the calculation carried out is useful in
an order-of-magnitude sense, further investigation of each of the simplifying
assumptions made is warranted.
In addition, the composition of bulk Earth has some uncertainties.
For example, the reported bulk Earth concentration of the siderphile element
\elem{Mn}, varies from $800$ to $1700$~ppm (\citealt{1998psc..book.....L}).
While the latter value from \citealt{mcdonough2001composition} has been used in
our calculation, the former value would bring the observed $\Delta\elemH{Mn}$
to an even closer agreement.
Given these limitations, the level of agreement for $\Delta\elemH{X}$ {\it and}
\elem{Li} for \bizarreone\ is remarkable.

\section{Summary}
\label{sec:summary}

We report the discovery of a comoving pair of bright
solar-type stars HD~240430 and HD~240429 (G0 and G2) with very different
metallicities ($\Delta\feh \approx 0.2$~dex), and condensation temperature
(\Tcondens)-dependent abundance differences.
The more metal-rich of the two stars, HD~240430 (\bizarreone), shows enhancement in
all ten elements with $\Tcondens > 1200$~K including \elem{Fe}, while under
enhanced in the five elements, \elem{C}, \elem{N}, \elem{O}, \elem{Na}, and
\elem{Mn} with $\Tcondens < 1200$~K relative to HD~240429 (\sunanalog).
The two stars also have anomalously high surface \elem{Li} abundance compared
to their ages of $\sim 4$~Gyr.
We consider that the comoving pair may have formed from two stars of different
birth origins in a exchange scattering event, or that there may be chemical
inhomogeneity in the birth cloud, and find both unlikely.

In order to explain the $\Tcondens$-dependent enhancement and high \elem{Li}
abundance, we consider accretion of planetary material as the cause.
We argue that a recent accretion of $20$~\mearth\ of bulk Earth
composition to \bizarreone\ can explain the enhancement in both refractory
elements and lithium.
For \sunanalog\ which also has high surface \elem{Li} abundance given its age,
we put an upper limit of $\approx 4$~\mearth\ on the accreted mass
based on the \elem{Li} concentration of carbonaceous chondrites and bulk Earth.
While the case is not as clear as \bizarreone\ without a chemical reference star,
the range of abundances from volatile to refractory elements is
similar to that expected from $\approx 4$~\mearth\ accretion of bulk Earth.
It is unclear what triggered the accretions to both stars.
One possibility is that a fly-by field star may have
perturbed the planetary systems of both stars.

The two stars have not been included in any publicly released data from planet
search programs.
If both stars have accreted planetary material, it would be very interesting to
search for the existence and architectures of the planetary systems left
behind.

%TODO: Implications
%- chemical tagging, frequency is of concern
%- exoplaneteers looking for accretion signature: look for the most odd ones in
%the expected direction not just in binaries already known to host a planet.

\acknowledgements

We thank Andy Casey for bringing $^{6}\elem{Li}$ into our attention.
We thank Megan Bedell and Andy Casey for valuable discussions,
and Keith Hawkins, Nathan Leigh, and Josh Winn for comments
on the early version of the draft.
The Flatiron Institute is supported by the Simons Foundation.

% Gaia
This work has made use of data from the European Space Agency (ESA) mission
{\it Gaia} (\url{http://www.cosmos.esa.int/gaia}), processed by the {\it Gaia}
Data Processing and Analysis Consortium (DPAC,
\url{http://www.cosmos.esa.int/web/gaia/dpac/consortium}). Funding for the DPAC
has been provided by national institutions, in particular the institutions
participating in the {\it Gaia} Multilateral Agreement.
% 2MASS
This publication makes use of data products from the Two Micron All Sky Survey,
which is a joint project of the University of Massachusetts and the Infrared
Processing and Analysis Center/California Institute of Technology, funded by
the National Aeronautics and Space Administration and the National Science
Foundation.
% all-WISE
This publication makes use of data products from the Wide-field Infrared Survey
Explorer, which is a joint project of the University of California, Los
Angeles, and the Jet Propulsion Laboratory/California Institute of Technology,
funded by the National Aeronautics and Space Administration.

%TODO: corner.py
%TODO: data/code available
\software{
  %The data and code used in this project is available from
  %\url{https://github.com/smoh/KronosKrios} under the MIT open-source
  %software license.
  This research utilized:
  \texttt{Astropy} (\citealt{Astropy-Collaboration:2013}),
  \texttt{emcee} (\citealt{2013PASP..125..306F}),
  \texttt{IPython} (\citealt{Perez:2007}),
  \texttt{matplotlib} (\citealt{Hunter:2007}),
  \texttt{numpy} (\citealt{Van-der-Walt:2011}),
  and \texttt{pandas} (\citealt{pandas}).}

\bibliography{ref}

\end{document}
